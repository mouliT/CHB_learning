% ============================================================
%  DC-Link Voltage Balancing Modulation for Cascaded H-Bridge Converters
%  Paper 3 Reference Notes
%  Source: [Vasishta et al.], IEEE Access
%  Figures stored in: figures/
%  Naming convention: DLVBMCHBC_Fig[N]_page[XX].png  (caption-extracted at 200 DPI)
% ============================================================
\documentclass[12pt, a4paper]{article}

% ---- core packages ----
\usepackage[utf8]{inputenc}
\usepackage[T1]{fontenc}
\usepackage{amsmath, amssymb, amsthm}
\usepackage{bm}
\usepackage{siunitx}

% ---- figures ----
\usepackage{graphicx}
\graphicspath{{figures/}}
\usepackage{float}
\usepackage{caption}
\usepackage{subcaption}

% ---- layout ----
\usepackage[a4paper, top=2.5cm, bottom=2.5cm, left=3cm, right=2.8cm]{geometry}
\usepackage{parskip}
\setlength{\parskip}{6pt}
\setlength{\parindent}{0pt}

% ---- hyperlinks ----
\usepackage[colorlinks=true, linkcolor=blue!60!black, citecolor=blue!50!black]{hyperref}
\usepackage{bookmark}
\makeatletter
\pdfstringdefDisableCommands{%
    \def\vspace#1{}%
    \def\rule#1#2{}%
    \def\color#1{}%
    \let\leavevmode@ifvmode\relax%
    \def\,{ }%
}
\makeatother

% ---- styled boxes ----
\usepackage{tcolorbox}
\tcbuselibrary{skins, breakable}

% ---- tables ----
\usepackage{booktabs}
\usepackage{array}
\usepackage{longtable}

% ---- section formatting ----
\usepackage{titlesec}
\usepackage{xcolor}
\definecolor{chbblue}{RGB}{10, 55, 120}
\definecolor{chborange}{RGB}{190, 70, 5}
\definecolor{chbgreen}{RGB}{8, 95, 38}
\definecolor{chbred}{RGB}{170, 25, 25}
\definecolor{lightblue}{RGB}{216, 232, 252}
\definecolor{lightorange}{RGB}{255, 244, 220}
\definecolor{lightgreen}{RGB}{216, 247, 228}
\definecolor{lightred}{RGB}{255, 228, 228}
\definecolor{lightgray}{RGB}{245, 245, 245}

\titleformat{\section}{\large\bfseries\color{chbblue}}{{\thesection.}}{0.5em}{}
            [\vspace{-4pt}\rule{\textwidth}{0.5pt}]
\titleformat{\subsection}{\normalsize\bfseries\color{chbblue!80!black}}{{\thesubsection}}{0.5em}{}

% ---- tcolorbox styles ----
\newtcolorbox{circuitbox}[1][Circuit Description]{
    colback=lightblue, colframe=chbblue, coltitle=white,
    fonttitle=\bfseries\small, title={#1},
    breakable, enhanced, left=4pt, right=4pt, top=4pt, bottom=4pt}

\newtcolorbox{statebox}[1][State Variables \& Parameters]{
    colback=lightorange, colframe=chborange, coltitle=white,
    fonttitle=\bfseries\small, title={#1},
    breakable, enhanced, left=4pt, right=4pt, top=4pt, bottom=4pt}

\newtcolorbox{eqnbox}[1][Governing Equations]{
    colback=lightgreen, colframe=chbgreen, coltitle=white,
    fonttitle=\bfseries\small, title={#1},
    breakable, enhanced, left=4pt, right=4pt, top=4pt, bottom=4pt}

\newtcolorbox{physbox}[1][Physical Interpretation]{
    colback=lightgray, colframe=gray!60, coltitle=white,
    fonttitle=\bfseries\small, title={#1},
    breakable, enhanced, left=4pt, right=4pt, top=4pt, bottom=4pt}

\newtcolorbox{warningbox}[1][Watch Out]{
    colback=lightred, colframe=chbred, coltitle=white,
    fonttitle=\bfseries\small, title={#1},
    breakable, enhanced, left=4pt, right=4pt, top=4pt, bottom=4pt}

% ---- custom commands ----
\newcommand{\vg}{v_{g}}
\newcommand{\Vg}{V_{g}}
\newcommand{\ig}{i_{g}}
\newcommand{\Ig}{I_{g}}
\newcommand{\omg}{\omega}
\newcommand{\phig}{\phi_{g}}
\newcommand{\Vdc}{V_{dc}}
\newcommand{\Vdci}[1]{V_{dc,#1}}
\newcommand{\Vtot}{V_{dc,\mathrm{tot}}}
\newcommand{\Mi}{M_{i}}
\newcommand{\phic}{\varphi}
\newcommand{\vref}{v_{\mathrm{ref}}}
\newcommand{\PH}{P_{H}}
\newcommand{\PHi}[1]{P_{H,#1}}
\newcommand{\PHnc}{P_{H}^{\mathrm{nc}}}
\newcommand{\PHcl}{P_{H}^{\mathrm{cl}}}
\newcommand{\figref}[1]{Figure~\ref{#1}}

% ---- figure entry macro ----
\newcommand{\figentry}[4]{%
    \begin{figure}[H]
        \centering
        \includegraphics[width=#4\textwidth]{#2}
        \caption{#3}
        \label{#1}
    \end{figure}%
}

% ============================================================
\title{\textbf{DC-Link Voltage Balancing Modulation}\\[4pt]
       \textbf{for Cascaded H-Bridge Converters}\\[6pt]
       \large Paper 3 --- Circuit Topology, Control Architecture,\\
       \large Loading Power Theory, and Sorting Algorithm\\[4pt]
       \normalsize IEEE Access}
\date{\today}
% ============================================================

\begin{document}
\maketitle
\tableofcontents
\clearpage

% ============================================================
\setcounter{section}{-1}
\section{Before You Begin --- The Big Picture}
\label{sec:bigpicture}
% ============================================================

\subsection{What Is This Document?}

This document is a self-contained reference guide to the paper:

\begin{center}
\textit{``DC-Link Voltage Balancing Modulation for Cascaded H-Bridge Converters''}\\
Y.\,Ko, A.\,Tcai, and M.\,Liserre, \textit{IEEE Access}, 2021.
\end{center}

It walks through every figure, every equation, and every design decision in
Feynman style --- starting from the simplest possible picture and building up
to the full algorithm. No derivation is skipped; no equation is left unexplained.

\textbf{What you will understand by the end:}
\begin{itemize}
    \item Why a Cascaded H-Bridge (CHB) inverter needs DC-link voltage balancing.
    \item What ``loading power'' is and why it is the right quantity to control.
    \item Why clamped modulation is faster than conventional PI control for balancing.
    \item How to compute the exact modulation parameters to balance any imbalance
          in minimum time.
\end{itemize}

\subsection{The Simplest Possible Description of a CHB}

\textbf{Start with a single H-bridge cell.} It has two ports:

\begin{itemize}
    \item \textbf{DC input port:} an isolated DC-link capacitor at voltage
          $\Vdci{i}$, continuously replenished by a DC source (a PV string,
          a battery pack, or a rectifier) delivering power $P_{iH}$.
          This port does not connect electrically to any other cell.
    \item \textbf{AC output port:} two terminals that the four transistors
          switch at high frequency. The average voltage across these terminals
          over one switching period is $v_i(t) = \Vdci{i} \cdot \Mi\sin(\omg t)$,
          where $\Mi$ is the modulation index the controller sets.
\end{itemize}

The H-bridge is therefore a controlled DC-to-AC converter: power flows in
from an isolated DC source and flows out as a shaped AC voltage.

\medskip
\textbf{Where exactly does cascading happen?}
In this paper, $n$ such cells are cascaded by connecting their \emph{AC output
terminals} in series --- the top terminal of cell~$i$ joins the bottom terminal
of cell~$(i{+}1)$, forming a single series string. The bottom terminal of the
lowest cell and the top terminal of the highest cell are the two ends of this
string. One end connects to the grid through filter inductance $L_f$; the other
end is the neutral. The same grid current $\ig$ is forced through every cell
in the string. The DC input ports play no part in the cascading --- they remain
fully isolated from one another throughout.

\medskip
The total AC voltage presented to the grid is the series sum of all cell outputs:
\[
    \vg(t) = v_1(t) + v_2(t) + \cdots + v_n(t)
\]

In sinusoidal PWM, each cell's transistors switch at a carrier frequency
$f_c \gg f$. The instantaneous output of cell $i$ is therefore a PWM waveform
that alternates rapidly between $+\Vdci{i}$, $0$, and $-\Vdci{i}$ --- not a
sinusoid. Averaged over one switching period, however, the \textbf{fundamental
component} of cell $i$'s output is:
\[
    \bar{v}_i(t) = \Vdci{i} \cdot \Mi \cdot \sin(\omg t)
\]

Summing across all $n$ cells, the fundamental component of the total output is:
\[
    \bar{\vg}(t) = \left(\sum_{i=1}^{n} \Vdci{i}\,\Mi\right)\sin(\omg t)
\]

\begin{warningbox}[Three Caveats on This Equation]
\begin{enumerate}
    \item \textbf{Fundamental only.} The actual instantaneous $\vg(t)$ contains
          high-frequency PWM harmonics on top of the sine. These are attenuated
          --- not eliminated --- by the filter inductance $L_f$.
    \item \textbf{No per-cell phase offset.} All cells in this paper share the
          same sinusoidal reference phase. There is no $\theta_i$ offset between
          cells in the voltage sum.
    \item \textbf{Linear modulation only.} The relation $\bar{v}_i = \Vdci{i}\Mi\sin(\omg t)$
          holds only for $\Mi \leq 1$ (non-clamped, linear region). Clamped
          modulation --- the key contribution of this paper --- modifies this,
          which is why later sections treat the clamped case separately.
\end{enumerate}
\end{warningbox}

\textbf{Why bother stacking?} Each cell's capacitor only needs to support a
fraction $\Vdci{i}$ of the total bus voltage, so low-voltage, low-cost
transistors suffice. A 540\,V grid interface is built from three 180\,V cells.
With sinusoidal PWM and phase-shifted carriers, $n$ cells together produce a
$(2n{+}1)$-level output; more levels mean lower harmonic distortion and a
smaller filter $L_f$ --- see \figref{fig:staircase_buildup}.

% TODO: Add a side-by-side n=1..4 comparison figure here (Figure 1).
%       Requirements: all panels on a normalised y-axis (v / n*Vdc) so the
%       shrinking orange error region is visually obvious across panels.
%       Defer until a clean Python figure meeting these criteria is ready.

\figentry{fig:staircase_buildup}{CHB_staircase_buildup}%
    {How three individual H-bridge cell outputs combine to form a
     seven-level multilevel waveform ($n = 3$, $f_c = 12f$, $M = 0.95$).
     \textbf{Rows 1--3:} each cell's unipolar PWM output (solid coloured
     line) is shown together with its triangular carrier (grey) and the
     shared sinusoidal reference $MV_{dc}\sin(\omega t)$ (dashed).
     Carriers are phase-shifted by $120^\circ$ so that the three cells
     switch at interleaved instants.
     \textbf{Row 4 (Sum):} the series sum $v_g = v_1 + v_2 + v_3$
     steps through seven discrete levels
     ($0,\,\pm V_{dc},\,\pm 2V_{dc},\,\pm 3V_{dc}$) and closely tracks
     the ideal sinusoid $3V_{dc}M\sin(\omega t)$ (red dashed line).
     The orange shading is the residual instantaneous error; the
     phase-shifted carriers ensure that no two cells switch at the same
     instant, spreading the transitions evenly and minimising the error.}{1.0}

\subsection{The Problem in One Sentence}

\textbf{Every cell shares the same AC current $\ig$, but each cell has its own
DC source delivering a different amount of power.}

If those source powers are unequal and nothing corrects it, the DC voltages
$\Vdci{i}$ drift apart. The cell with more input power charges up; the cell
with less power drains down. Left uncorrected, one cell charges to destruction
while another discharges to zero.

\medskip
This is the \emph{normal operating condition} for a CHB connected to:
\begin{itemize}
    \item \textbf{Photovoltaic strings}: different panels see different shading.
    \item \textbf{Battery packs}: cells age at different rates and reach different
          states of charge.
    \item \textbf{Power routing systems}: cells deliberately assigned unequal power
          to extend component lifetime.
\end{itemize}

Balancing is not an optional add-on --- it is a hard requirement for safe, stable
operation.

\subsection{What This Paper Proposes --- In One Paragraph}

The conventional fix is a per-cell PI controller that nudges each cell's
modulation index by a small $\Delta M_i$ each cycle. It works, but the measured
settling time is $\approx 1850$\,ms (92 grid cycles at 50\,Hz). The PI can
only make a small correction per cycle, otherwise it distorts the grid current.

\medskip
This paper asks a different question: \emph{what is the maximum loading power
this cell can absorb or deliver, given its modulation strategy?} It then
\textbf{directly sets} the modulation parameters $M_i$ and clamping angle
$\phic$ to deliver exactly that maximum loading power from cycle one.
The result: $\approx 140$\,ms settling time --- a factor of $13\times$ faster.

\begin{eqnbox}[The One Number That Matters]
\[
    \boxed{13\times \text{ faster:} \quad 1850\,\text{ms}
           \;\longrightarrow\; 140\,\text{ms}}
\]
Both methods reach the same steady-state result --- balanced DC-link voltages
and sinusoidal grid current. The only difference is \emph{how fast they get
there}.
\end{eqnbox}

\subsection{Prerequisites}

This document assumes you are comfortable with:
\begin{enumerate}
    \item \textbf{Basic switching converters}: duty cycle, average voltage, PWM.
          If you understand what a buck converter does, you are ready.
    \item \textbf{AC circuit analysis}: phasors, power factor, the formula
          $P = V I \cos\theta$.
    \item \textbf{PI controllers}: a proportional-integral controller integrates
          the error and adjusts its output. That is all.
\end{enumerate}

You do \emph{not} need: advanced state-space control theory, transistor-level
circuit simulation, or prior knowledge of CHB converters.

\subsection{Quick Symbol Reference}

Every symbol used in this document is defined here. Return to this table whenever
an equation introduces unfamiliar notation.

\begin{statebox}[Master Symbol Table]
\renewcommand{\arraystretch}{1.3}
\begin{tabular}{p{0.17\linewidth} p{0.40\linewidth} p{0.35\linewidth}}
\toprule
\textbf{Symbol} & \textbf{Meaning} & \textbf{Typical Value / Units} \\
\midrule
$n$                    & Number of H-bridge cells            & 3 (prototype), 4 (analysis) \\
$\Vdci{i}$            & DC-link voltage, cell $i$           & 180\,V (target each) \\
$\Vtot$                & Total DC voltage $\sum_i \Vdci{i}$ & $n \times 180$\,V \\
$\ig$, $\Ig$          & Grid current (inst.\ / amplitude)   & A; same through all cells \\
$\omg$                 & Grid angular frequency              & $2\pi \times 50$\,rad/s \\
$\phig$                & Grid power factor angle             & Fixed by load \\
$L_f$                  & Grid-side filter inductance         & mH \\
$\Mi$                  & Modulation index, cell $i$          & $0 \leq M_i \leq 1$ \\
$\phic$                & Clamping angle                      & $0$ (none) to $\pi/2$ (full) \\
$\PH$                  & Loading power (net into cap)        & W \\
$\PHi{i}$             & Loading power, cell $i$             & W \\
$\PHnc$                & Non-clamped loading power           & fn.\ of $\Mi$ only \\
$\PHcl$                & Clamped loading power               & fn.\ of $\Mi$ and $\phic$ \\
$P_{i,\mathrm{src}}$  & DC source power, cell $i$           & W; unequal between cells \\
\bottomrule
\end{tabular}
\end{statebox}

\clearpage

% ============================================================
\section{The Circuit --- What Is Actually Connected to What?}
\label{sec:circuit}
% ============================================================

Section~0 described the CHB in words. This section puts real components
on the page and names every wire, so that later equations have something
concrete to refer to.

% ── Fig 1 from the paper ──────────────────────────────────────────────────
\figentry{fig:paper_fig1}{DLVBMCHBC_Fig1_page01}%
    {Grid-connected CHB converter with unbalanced loading power
     $\PHi{1} \neq \PHi{2} \neq \PHi{3}$ (paper Fig.\,1).
     Each H-bridge cell has its own isolated DC-link capacitor fed by an
     independent DC source. The AC output terminals of all $n$ cells are
     connected in series; the series string drives the grid through
     filter inductance $L_f$.}{0.75}

\subsection{Reading the Circuit Diagram}

\figref{fig:paper_fig1} shows the complete system. Start at the bottom
and work upward.

\begin{circuitbox}[Component-by-Component Walk-through]
\begin{enumerate}
    \item \textbf{DC sources (right side of each cell).}
          Each cell $i$ has its own private DC source --- a PV string,
          battery pack, or rectifier --- delivering power $P_{i,\mathrm{src}}$
          and maintaining a DC-link capacitor at voltage $\Vdci{i}$.
          The dashed isolation boundary in the figure is not decorative:
          there is \emph{no} electrical path between the DC sides of
          different cells.

    \item \textbf{H-bridge switches (centre of each cell).}
          Four transistors per cell chop the DC voltage at carrier
          frequency $f_c$. The average AC voltage produced at the
          output terminals over one switching period is
          $\bar{v}_i = \Vdci{i}\Mi\sin(\omg t)$.

    \item \textbf{AC series string (left column).}
          The AC output terminals are stacked in series:
          the top terminal of cell~$i$ connects to the bottom terminal
          of cell~$(i{+}1)$. The total voltage across the string is
          $\vg = v_1 + v_2 + \cdots + v_n$.

    \item \textbf{Filter inductor $L_f$.}
          Sits between the AC string and the grid. It attenuates
          the high-frequency PWM harmonics and forces the grid
          current $\ig$ to be approximately sinusoidal.
          Crucially, \emph{the same} $\ig$ flows through every cell
          in the series string --- there is no way for current to
          take a different path.

    \item \textbf{Grid (far left).}
          Modelled as an ideal sinusoidal voltage source
          $v_{\mathrm{grid}}(t) = V_{\mathrm{grid}}\sin(\omg t + \phig)$.
          The grid sets the frequency $\omg$ and the power-factor
          angle $\phig$; the CHB controls the amplitude and phase of
          $\vg$ to regulate $\ig$.
\end{enumerate}
\end{circuitbox}

\subsection{The One Quantity That Drives Everything: Loading Power}
\label{subsec:loading_power}

Because the same current $\ig$ passes through every cell, each cell
\emph{must} process its share of the AC power whether it likes it or
not. But each cell also has its own DC source pushing power in from
the right. The net power that actually charges or discharges the
DC-link capacitor of cell $i$ is called its \textbf{loading power}:

\begin{eqnbox}[Definition of Loading Power]
\[
    \PHi{i} \;=\; \frac{1}{T}\int_{0}^{T} v_i(t)\,\ig(t)\,\mathrm{d}t
\]
where $T = 1/f$ is one grid period. This is the average AC power
\emph{delivered by cell $i$ to the grid} (positive = delivering,
negative = absorbing). In steady state, $\PHi{i}$ must equal the
DC source power $P_{i,\mathrm{src}}$ to keep $\Vdci{i}$ constant.
\end{eqnbox}

\begin{physbox}[Why ``Loading Power'' and Not Just ``Power''? --- The Governing Equation]
The DC-link capacitor of cell $i$ stores energy
$E_i = \tfrac{1}{2}C\Vdci{i}^2$.
Two power flows act on it simultaneously:
\begin{itemize}
    \item $P_{i,\mathrm{src}}$ flows \textbf{in} from the DC source.
    \item $\PHi{i}$ flows \textbf{out} to the AC series string.
\end{itemize}
The net rate of energy change is their difference, which by
$\tfrac{d}{dt}(\tfrac{1}{2}C V^2) = CV\dot{V}$ gives:
\[
    C\,\Vdci{i}\,\frac{d\Vdci{i}}{dt}
    \;=\; P_{i,\mathrm{src}} - \PHi{i}
\]
Three consequences follow directly from this one equation:
\begin{enumerate}
    \item \textbf{Steady state} ($\dot{V}_{dc,i} = 0$) requires exactly
          $\PHi{i} = P_{i,\mathrm{src}}$ --- the AC side must absorb
          precisely what the DC source supplies.
    \item \textbf{Overcharging:} if $P_{i,\mathrm{src}} > \PHi{i}$
          the RHS is positive $\Rightarrow$ $\Vdci{i}$ rises.
    \item \textbf{Discharging:} if $P_{i,\mathrm{src}} < \PHi{i}$
          the RHS is negative $\Rightarrow$ $\Vdci{i}$ falls.
\end{enumerate}
This is why the paper focuses entirely on controlling $\PHi{i}$:
it is the \emph{only} quantity the modulator can manipulate to
keep $\Vdci{i}$ at its target value.
\end{physbox}

\begin{warningbox}[The Key Asymmetry --- and What ``Continuously Corrects'' Means]
The grid current $\ig$ is \emph{shared} --- it is the same for every
cell. The DC source powers $P_{i,\mathrm{src}}$ are \emph{independent}
--- they depend on sunlight, battery state of charge, or deliberate
power routing, and can differ arbitrarily between cells and change
from one grid cycle to the next.

\medskip
From the capacitor equation derived above, the steady-state condition
for cell $i$ at grid cycle $k$ is:
\[
    \PHi{i}(k) \stackrel{!}{=} P_{i,\mathrm{src}}(k)
\]
Substituting the non-clamped loading power expression
(derived in Section~2):
\[
    \underbrace{\tfrac{1}{2}\,\Vdci{i}(k)\,\Mi(k)\,\Ig\cos\phig}_{P_{H,i}(k)}
    \;=\; P_{i,\mathrm{src}}(k)
\]
The controller must solve this for $\Mi(k)$ \emph{every grid cycle}
because $P_{i,\mathrm{src}}(k)$ changes continuously (cloud passing
over a PV panel, battery charging/discharging). That is what
``continuously corrects'' means: re-computing and re-applying $\Mi$
at each cycle $k$ to keep the left-hand side tracking the
right-hand side despite a moving target. This is the problem the
paper solves --- and does so $13\times$ faster than a PI controller.
\end{warningbox}

\clearpage

% ============================================================
\section{From Loading Power to Control}
\label{sec:loading_power_control}
% ============================================================

Section~1 introduced loading power $\PHi{i}$ as an integral and stated
that the controller must set it equal to $P_{i,\mathrm{src}}$ every
grid cycle.  Before designing any controller we need a \emph{closed-form
expression} for $\PHi{i}$ --- one that shows clearly which knob
($\Mi$, later $\phic$) the modulator can turn.

% ── §2.1 ─────────────────────────────────────────────────────────────────
\subsection{Evaluating the Loading Power Integral}
\label{subsec:PH_derivation}

The definition from Section~1 is:
\[
    \PHi{i} \;=\; \frac{1}{T}\int_{0}^{T} v_i(t)\,\ig(t)\,\mathrm{d}t
\]

\textbf{Step 1 — Write down the two signals.}

The fundamental component of cell $i$'s output voltage (averaged over
one switching period, linear modulation):
\[
    v_i(t) = \Vdci{i}\,\Mi\,\sin(\omg t)
\]

The grid current (sinusoidal, lagging the voltage reference by the
power-factor angle $\phig$):
\[
    \ig(t) = \Ig\,\sin(\omg t - \phig)
\]

\textbf{Step 2 — Substitute into the integral.}
\[
    \PHi{i}
    = \frac{\Vdci{i}\Mi\Ig}{T}
      \int_{0}^{T} \sin(\omg t)\,\sin(\omg t - \phig)\,\mathrm{d}t
\]

\textbf{Step 3 — Use the product-to-sum identity.}

$\sin A \sin B = \tfrac{1}{2}[\cos(A-B) - \cos(A+B)]$, so:
\[
    \sin(\omg t)\,\sin(\omg t - \phig)
    = \tfrac{1}{2}\bigl[\cos(\phig) - \cos(2\omg t - \phig)\bigr]
\]

\textbf{Step 4 — Integrate term by term.}

The $\cos(\phig)$ term is a constant; the $\cos(2\omg t - \phig)$ term
is a sinusoid at \emph{twice} the grid frequency, whose integral over
exactly one period $T = 2\pi/\omg$ is zero:
\[
    \frac{1}{T}\int_{0}^{T}\cos(2\omg t - \phig)\,\mathrm{d}t = 0
\]

\textbf{Result --- the non-clamped loading power:}

\begin{eqnbox}[Non-Clamped Loading Power $\PHnc$]
\[
    \boxed{
    \PHi{i}^{\,\mathrm{nc}}
    \;=\; \frac{1}{2}\,\Vdci{i}\,\Mi\,\Ig\,\cos\phig
    }
\]
\begin{itemize}
    \item $\Vdci{i}$ --- DC-link voltage of cell $i$ (state variable,
          measured each cycle).
    \item $\Mi$ --- modulation index of cell $i$
          (\textbf{the control knob}, $0 \leq \Mi \leq 1$).
    \item $\Ig$ --- grid current amplitude (measured, same for all cells).
    \item $\cos\phig$ --- grid power factor (fixed by the load/grid).
\end{itemize}
This is called \emph{non-clamped} because it assumes linear, unclipped
sinusoidal modulation.  When the paper introduces clamped modulation in
Section~4, a second free variable $\phic$ enters and $\PHi{i}$ takes a
different form.
\end{eqnbox}

\begin{physbox}[What the Formula Tells You]
Rearranging for $\Mi$ gives the required modulation index to achieve a
target loading power $\PHi{i}^{*}$:
\[
    \Mi^{*} = \frac{2\,\PHi{i}^{*}}{\Vdci{i}\,\Ig\,\cos\phig}
\]
The controller therefore needs to measure three quantities
($\Vdci{i}$, $\Ig$, $\cos\phig$) and know the target loading power
$\PHi{i}^{*} = P_{i,\mathrm{src}}$ to compute $\Mi^{*}$ directly ---
\emph{no iteration, no trial-and-error.}
The paper exploits this closed form to achieve one-cycle balancing.
\end{physbox}

% ── §2.2 ─────────────────────────────────────────────────────────────────
\subsection{The Overall Control Scheme}
\label{subsec:control_scheme}

\figentry{fig:paper_fig2}{DLVBMCHBC_Fig2_page02}%
    {Typical control scheme for the grid-connected CHB converter
     (paper Fig.\,2). An outer DC-link voltage control loop generates
     the grid-current reference $\ig^{*}$.  An inner current control
     loop tracks $\ig^{*}$ and produces the common modulation signal.
     A separate balancing algorithm adjusts each cell's individual
     modulation index $\Mi$ to keep all DC-link voltages equal at
     their target $\Vdci{} = \Vtot/n$.}{0.90}

\figref{fig:paper_fig2} shows the two-layer control structure used
throughout the paper.

\begin{circuitbox}[Two-Layer Control --- What Each Layer Does]
\begin{enumerate}
    \item \textbf{Outer loop --- DC-link voltage control.}
          This slower loop monitors the DC-link voltages $\Vdci{i}$
          (or their sum $\Vtot$) and compares them to the desired
          target.  Its output is the grid-current reference amplitude
          $\Ig^{*}$, which tells the inner loop how much current
          to draw from the grid.

    \item \textbf{Inner loop --- grid current control.}
          This faster loop (PR or PI controller) compares the measured
          grid current $\ig$ to the reference $\ig^{*}$ from the outer
          loop and outputs a common voltage reference
          $v_{\mathrm{ref}}(t) = V_{\mathrm{ref}}\sin(\omg t)$
          shared by all $n$ cells.
          Its job: keep $\ig$ sinusoidal and tracking $\ig^{*}$.

    \item \textbf{Balancing algorithm.}
          Running alongside the two feedback loops, the balancing
          algorithm computes an individual $\Mi$ for each cell so
          that $\PHi{i}$ tracks $P_{i,\mathrm{src}}$.
          The \emph{sum} $\sum_i \Vdci{i}\Mi$ is constrained to equal
          $V_{\mathrm{ref}}$, so the inner loop's modulation target
          is respected even while individual $\Mi$ values differ.
\end{enumerate}
\end{circuitbox}

\begin{statebox}[The Constraint the Balancing Algorithm Must Obey]
The inner current loop demands a total output voltage:
\[
    \vg(t) = \sum_{i=1}^{n} \Vdci{i}\,\Mi\,\sin(\omg t)
           = V_{\mathrm{ref}}\,\sin(\omg t)
\]
So any redistribution of $\Mi$ values by the balancing loop must
satisfy:
\[
    \sum_{i=1}^{n} \Vdci{i}\,\Mi \;=\; V_{\mathrm{ref}}
\]
This is the \emph{coupling constraint} between the two loops.
Balancing is not free --- every change to one cell's $\Mi$ forces
compensating changes in the others to keep the sum fixed.
\end{statebox}

\clearpage

% ══════════════════════════════════════════════════════════════════════════════
\section{Two Modulation Strategies --- The Paper's Key Idea}
\label{sec:two_modes}
% ══════════════════════════════════════════════════════════════════════════════

We established in \S\ref{sec:loading_power_control} that the balancing algorithm
adjusts each cell's modulation index $\Mi$ so that its loading power
$\PHi{i}$ matches its source power $P_{i,\mathrm{src}}$.  But \emph{how
much correction} can you actually achieve by varying $\Mi$?  This section
shows why a simple per-cell PI controller is limited, and how the paper
breaks that limit with a two-mode modulation strategy.

% ── §3.1 ─────────────────────────────────────────────────────────────────────
\subsection{The Conventional PI Approach and Its Limit}
\label{subsec:pi_limit}

\figentry{fig:paper_fig3}{DLVBMCHBC_Fig3_page02}%
    {Conventional PI loop-based voltage balancing algorithm
     (paper Fig.\,3).  Each cell $i$ has its own PI controller that
     compares the common DC-link target $V_{dc,H}^{*}$ to the measured
     voltage $V_{dc,iH}$ and produces a small correction
     $\Delta v_{iH}^{*}$.  This correction is added to the shared
     modulation reference $v^{*}$ to give each cell its individual
     voltage reference $v_{iH}^{*}$.}{0.72}

The conventional approach shown in \figref{fig:paper_fig3} is
straightforward: if cell $i$ is too high ($V_{dc,iH} > V_{dc,H}^{*}$),
the PI controller raises $\Delta v_{iH}^{*}$, which increases that
cell's effective modulation index and therefore its loading power,
drawing more energy out of its capacitor.

\begin{physbox}[Why PI Correction Has a Hard Ceiling]
The PI output $\Delta v_{iH}^{*}$ is simply added to a sinusoidal
reference $v^{*} = V_{\mathrm{ref}}\sin(\omg t)$.  The resulting
cell reference is still sinusoidal in shape:
\[
    v_{iH}^{*}(t) \;=\; (V_{\mathrm{ref}} + \Delta v_{iH}^{*})\,\sin(\omg t)
\]
The modulation index is bounded: $M_i \leq 1$ (over-modulation causes
waveform distortion).  So the maximum extra loading power the PI method
can deliver is set by how much headroom remains between the current
$M_i$ and 1.  At high nominal modulation depths (e.g.\ $M = 0.9$),
that headroom is only 10\,\% --- very little room to push a heavily
imbalanced cell back in line.
\end{physbox}

This hard ceiling on achievable correction power is exactly what
paper Fig.\,9 (covered in a later section) will confirm
quantitatively.  The question is: can we give the balancing algorithm
\emph{more} correction range without violating the modulation limit?

% ── §3.2 ─────────────────────────────────────────────────────────────────────
\subsection{Clamped Modulation --- A New Degree of Freedom}
\label{subsec:clamped_mode}

% ·· Level-1 intuition ·····················································
\begin{physbox}[Start here: what happens to $v_i \cdot \ig$ near a zero crossing?]
Loading power is the area under the curve $p(\theta) = v_i(\theta)\cdot\ig(\theta)$.
Ask yourself: \emph{how much area does the region near a zero crossing contribute?}

\textbf{Non-clamped case.}
At unity power factor both $v_i$ and $\ig$ are sinusoids at the same
frequency, so they reach zero \emph{at the same moment}.
Near the crossing at $\theta = 0$:
\[
  v_{nc} \approx M\Vdc\cdot\theta, \qquad
  \ig    \approx \Ig\cdot\theta
  \qquad\Longrightarrow\qquad
  p(\theta) = v_{nc}\cdot\ig \approx M\Vdc\Ig\,\theta^{2}
\]
The power near the crossing goes as $\theta^2$ --- a \emph{parabola} that
pinches tightly to zero.  The area of a thin strip of width $\phic$ under a
parabola is $\propto \phic^{3}/3$, which is extremely small for small $\phic$.

\textbf{Clamped case.}
Now $v_c^{*}$ is frozen at $M\Vdc\sin\phic$ while $\ig$ is still a sinusoid:
\[
  v_c^{*} = M\Vdc\sin\phic \;(\text{constant}), \qquad
  \ig \approx \Ig\cdot\theta
  \qquad\Longrightarrow\qquad
  p(\theta) = M\Vdc\Ig\,\sin\phic\cdot\theta
\]
Now the power near the crossing goes as $\theta$ --- a \emph{straight line},
not a parabola.  The area under a straight line of width $\phic$ is
$\propto \phic^{2}/2$, which is \emph{much larger} than $\phic^{3}/3$ for
small $\phic$.

\textbf{The key insight:}
Non-clamped suffers from a \emph{double penalty} --- both $v$ and $i$ go
to zero together, so their product goes to zero as $\theta^2$.
Clamping breaks this: $v$ stays nonzero, so the product goes to zero only
as $\theta^1$.  The near-crossing region contributes real area to the
loading power integral instead of contributing almost nothing.
Wider $\phic$ means a larger near-crossing window converts from
``parabolic-zero'' to ``linear'', so more total area is gained.
\end{physbox}

% ·· Mode 1 recap ··························································
\noindent\textbf{Mode 1 --- Non-clamped ($\PHnc$): standard sinusoid.}

The reference is $v_{nc}^{*}(\theta) = M\sin\theta$ for the entire cycle.
Loading power (derived in \S\ref{subsec:PH_derivation}):
\[
    \PHnc \;=\; \tfrac{1}{2}\,\Vdc\,M\,\Ig\,\cos\phig
\]
Once $M$ and $\Ig$ are fixed by the control loops, $\PHnc$ is locked.
\textbf{No free parameter remains.}

% ·· Mode 2 definition ·····················································
\bigskip
\noindent\textbf{Mode 2 --- Clamped ($\PHcl$): frozen windows near zero
crossings.}

\figentry{fig:paper_fig7}{DLVBMCHBC_Fig7_page04}%
    {Clamped modulation signal (paper Fig.\,7).  The solid portions
     follow the sinusoid near the peak.  The dotted flat segments are
     the \emph{clamped} regions: the reference is frozen at
     $\pm M\sin\phic$ for a window of half-width $\phic$ either side
     of each zero crossing.  Larger $\phic$ widens the flat region.}{0.62}

The clamped reference is a piecewise function.  Writing $\theta = \omg t
\in [0, 2\pi)$ and using the clamping half-angle $\phic \in [0,\pi/2]$:

\begin{eqnbox}[Piecewise definition of $v_c^{*}(\theta)$]
\[
v_c^{*}(\theta) \;=\; M \cdot
\begin{cases}
  \sin\phic          & \theta \in [0,\,\phic) \\[4pt]
  \sin\theta         & \theta \in [\phic,\;\pi-\phic] \\[4pt]
  \sin\phic          & \theta \in (\pi-\phic,\;\pi) \\[4pt]
  -\sin\phic         & \theta \in [\pi,\;\pi+\phic) \\[4pt]
  \sin\theta         & \theta \in [\pi+\phic,\;2\pi-\phic] \\[4pt]
  -\sin\phic         & \theta \in (2\pi-\phic,\;2\pi]
\end{cases}
\]
\textbf{Sanity check:} at $\phic = 0$ every flat piece has
$\sin 0 = 0$ and the sinusoidal pieces cover the full cycle
$\Rightarrow$ $v_c^{*} = M\sin\theta$ = non-clamped.
At $\phic = \pi/2$ the sinusoidal pieces vanish and
$v_c^{*} = M\,\mathrm{sign}(\sin\theta)$ --- a square wave of amplitude $M$.
\end{eqnbox}

\begin{circuitbox}[Reading the piecewise formula piece by piece]
\begin{enumerate}
  \item $[0, \phic)$: the cycle has just started (positive half),
        but the reference is already \emph{frozen} at $M\sin\phic$
        --- a small positive constant --- instead of rising from zero.

  \item $[\phic, \pi-\phic]$: the reference tracks the sinusoid
        normally, sweeping through its positive peak at $\theta=\pi/2$.
        This is the only sinusoidal region in the positive half-cycle.

  \item $(\pi-\phic, \pi)$: approaching the zero crossing at $\pi$,
        the reference is frozen again at $M\sin\phic$ rather than
        descending to zero.  The signal does \emph{not} cross zero
        smoothly; it makes a step down to $-M\sin\phic$ at $\theta=\pi$.

  \item $[\pi, \pi+\phic)$: the negative half-cycle starts with the
        reference frozen at $-M\sin\phic$ (mirror of piece 1).

  \item $[\pi+\phic, 2\pi-\phic]$: sinusoidal region of the negative
        half-cycle, sweeping through the trough at $\theta = 3\pi/2$.

  \item $(2\pi-\phic, 2\pi]$: frozen at $-M\sin\phic$ before the next
        positive zero crossing (mirror of piece 3).
\end{enumerate}
\end{circuitbox}

% ·· Why it increases loading power ········································
\subsubsection*{Why does clamping increase the loading power?}

The loading power is $P = (1/T)\!\int_0^T v_i\,\ig\,dt$.  With $\theta =
\omg t$ and unity power factor ($\phig = 0$) so that $\ig = \Ig\sin\theta$:

\[
  \PHcl
  \;=\;
  \frac{M\Vdc\Ig}{2\pi}
  \int_0^{2\pi} f_c(\theta)\,\sin\theta\;d\theta
\]

where $f_c(\theta)$ is the normalised clamped waveform
($v_c^{*} = M\Vdc\,f_c$).  Breaking the integral over the six pieces
above:

\[
  \int_0^{2\pi} f_c\,\sin\theta\;d\theta
  \;=\;
  \underbrace{2\!\int_{\phic}^{\pi-\phic}\!\sin^2\!\theta\;d\theta}_{%
    \text{sinusoidal pieces (I, V)}}
  \;+\;
  \underbrace{4\sin\phic\!\int_0^{\phic}\!\sin\theta\;d\theta}_{%
    \text{flat pieces (II, III, IV, VI)}}
\]

Evaluating each group:
\begin{align*}
  2\!\int_{\phic}^{\pi-\phic}\!\sin^2\!\theta\;d\theta
    &= (\pi - 2\phic) + \sin 2\phic \\[4pt]
  4\sin\phic\!\int_0^{\phic}\!\sin\theta\;d\theta
    &= 4\sin\phic\,(1 - \cos\phic)
\end{align*}

\begin{eqnbox}[Clamped loading power --- closed form]
\[
  \boxed{
    \PHcl
    \;=\;
    \frac{M\Vdc\Ig\cos\phig}{2\pi}
    \Bigl[(\pi - 2\phic) + \sin 2\phic + 4\sin\phic(1-\cos\phic)\Bigr]
  }
\]
(The $\cos\phig$ factor reinstates general power factor by symmetry of
the derivation.)  Dividing by $\PHnc = \tfrac{1}{2}M\Vdc\Ig\cos\phig$:
\[
  \frac{\PHcl}{\PHnc}
  \;=\;
  \frac{(\pi - 2\phic) + \sin 2\phic + 4\sin\phic(1-\cos\phic)}{\pi}
\]
\end{eqnbox}

\begin{statebox}[Numerical values of the power ratio at unity power factor]
\centering
\begin{tabular}{ccc}
\hline
$\phic$ & $\PHcl/\PHnc$ & Comment \\
\hline
$0^{\circ}$   & 1.000 & No clamping --- identical to non-clamped \\
$30^{\circ}$  & 1.027 & Barely visible flat region \\
$60^{\circ}$  & 1.161 & Noticeable flat region near crossings \\
$90^{\circ}$  & $4/\pi \approx 1.273$ & Square wave --- maximum at unity PF \\
\hline
\end{tabular}

\smallskip
\raggedright
At $\phic = 90^{\circ}$ the clamped signal becomes a square wave
($v_c^{*} = M\Vdc\,\mathrm{sign}(\sin\theta)$), whose fundamental
Fourier coefficient is $4/\pi$ --- the theoretical ceiling at unity PF.
Paper Fig.\,8 shows values up to ${\approx}1.8$\,p.u.\ at non-unity
power factor angles, where the asymmetry between $v_c^{*}$ and $\ig$
zero-crossings provides additional area in the integral.
\end{statebox}

% ·· Fourier argument ······················································
\subsubsection*{Why is reshaping the waveform acceptable?}

The clamped waveform is \emph{not} a pure sinusoid, so why does the grid
not object?  The Fourier series answers this.

\begin{physbox}[Fourier view of the clamped signal]
Any periodic odd waveform with half-wave symmetry expands as:
\[
  v_c^{*}(\theta)
  \;=\;
  \sum_{\substack{n=1,3,5,\ldots}} b_n \sin(n\theta)
\]
Only odd harmonics appear.  The fundamental coefficient is the $b_1$
we already computed:
\[
  b_1 = \frac{M}{\pi}\bigl[(\pi-2\phic) + \sin 2\phic
        + 4\sin\phic(1-\cos\phic)\bigr] > M \quad\text{for } \phic > 0
\]
Higher harmonics ($n = 3, 5, 7, \ldots$) are present but attenuated:
\begin{itemize}
  \item The $n$-th harmonic has voltage amplitude $\sim b_n \Vdc$.
  \item The filter inductance $L_f$ presents impedance $n\omg L_f$ to
        the $n$-th harmonic.
  \item The resulting harmonic current $\sim b_n\Vdc / (n\omg L_f)$
        falls off as $1/n$.
  \item Harmonic power $\sim (b_n)^2/n^2$ falls off as $1/n^2$.
\end{itemize}
At $\phic = 90^{\circ}$ (square wave) the spectrum is $b_n = 4M/n\pi$,
so the 3rd harmonic current is $1/3$ and its power is $1/9$ of the
fundamental.  The H-bridge already produces PWM switching harmonics at
$f_c = 12f$ (and multiples) that $L_f$ is designed to block.  The
clamping harmonics at $3f, 5f, \ldots$ lie well below $f_c$ but are
also suppressed by $L_f$, keeping the grid-current THD within code limits.

\textbf{Crucially}: only the fundamental $b_1\sin\theta$ overlaps with
$\ig = \Ig\sin\theta$ and therefore contributes to real power transfer.
The harmonics are reactive and filtered.  The balancing algorithm is
exploiting exactly this: it raises $b_1 > M$ by clamping, thereby
transferring more real power per cycle, without needing to raise $M$
above its saturation limit of 1.
\end{physbox}

\begin{warningbox}[$\phic$ and $M$ are independent knobs]
Changing $M$ scales every point of the sinusoid uniformly --- the shape
is unchanged, only the amplitude changes.  Changing $\phic$ \emph{reshapes}
the waveform: the peak amplitude $M$ is untouched, but the flat windows
near the zero crossings grow wider.  In the paper, $M$ is fixed globally
by the outer voltage and inner current loops; $\phic$ is what the
balancing algorithm tunes \emph{independently per cell}.
\end{warningbox}

% ── §3.3 ─────────────────────────────────────────────────────────────────────
\subsection{How Clamping Changes Loading Power}
\label{subsec:loading_power_surfaces}

\figentry{fig:paper_fig8}{DLVBMCHBC_Fig8_page04}%
    {Loading power as a function of nominal modulation index $M$ and
     clamping angle $\varphi$ (paper Fig.\,8).
     (a)~Clamped mode: the loading power surface rises steeply,
     reaching up to ${\approx}1.8$\,p.u.\ at small $\varphi$.
     (b)~Non-clamped mode: the surface is nearly flat at
     ${\approx}1$\,p.u.\ across all $M$ and $\varphi$ --- no tuning
     knob.}{0.88}

\figref{fig:paper_fig8} is the quantitative payoff.  Compare the two
panels side by side:

\begin{statebox}[Reading the Loading Power Surfaces]
\begin{itemize}
    \item \textbf{Non-clamped (right panel):} the surface is almost
          perfectly flat at 1\,p.u.\ regardless of $M$ or $\varphi$.
          Once $M$ is set by the current controller, the loading power
          is fixed.  The balancing algorithm has \emph{no room to
          manoeuvre}.

    \item \textbf{Clamped (left panel):} the surface climbs from
          1\,p.u.\ (large $\varphi$, i.e.\ heavily clamped) up to
          ${\approx}1.8$\,p.u.\ (small $\varphi$, i.e.\ barely
          clamped).  By choosing $\varphi$, the algorithm can push a
          cell's loading power \emph{up to 80\,\% above the baseline} ---
          without ever exceeding $M = 1$.

    \item \textbf{The balancing idea in one sentence:}
          assign clamped mode to cells that need to absorb \emph{more}
          power (overcharged capacitors), and non-clamped mode to cells
          that should absorb \emph{less} (undercharged or on-target).
          The next section shows exactly how the algorithm decides
          which cell gets which mode.
\end{itemize}
\end{statebox}

% ── §3.4 ─────────────────────────────────────────────────────────────────────
\subsection{Overview: How the Two Modes Are Assigned}
\label{subsec:algorithm_overview}

\figentry{fig:paper_fig4}{DLVBMCHBC_Fig4_page03}%
    {Principle of the proposed method (paper Fig.\,4).  All measured
     DC-link voltages $V_{dc,iH}$ are fed into a sorting algorithm
     together with the reference $V_{dc,H}^{*}$.  The sorted list
     places the reference somewhere in the middle.  Cells ranked
     above the reference receive one modulation mode; cells ranked
     below receive the other.  Which mode goes to which group flips
     depending on the sign of $\vg\cdot\ig$ (power flow direction).}{0.72}

Now that we have the two tools (non-clamped and clamped), the question
is: which cell gets which mode at any given moment?
\figref{fig:paper_fig4} shows the answer in one diagram.

\begin{circuitbox}[The Assignment Rule in Three Steps]
\begin{enumerate}
    \item \textbf{Measure.}  Sample all DC-link voltages
          $V_{dc,1H},\,V_{dc,2H},\,\ldots,\,V_{dc,nH}$ and the
          reference $V_{dc,H}^{*}$.

    \item \textbf{Sort.}  Rank all voltages in descending order and
          locate where $V_{dc,H}^{*}$ falls in the list:
          \[
              V_{dc,1H} \;>\; V_{dc,2H} \;>\; \cdots \;>\;
              \underbrace{V_{dc,H}^{*}}_{\text{reference}} \;>\;
              \cdots \;>\; V_{dc,nH}
          \]
          Cells \emph{above} the reference are overcharged
          ($V_{dc,iH} > V_{dc,H}^{*}$) and need to shed more energy.
          Cells \emph{below} are undercharged or on-target.

    \item \textbf{Assign.}  The sign of the instantaneous power
          $\vg\cdot\ig$ tells us whether the grid is absorbing or
          delivering energy right now:
          \begin{itemize}
              \item If $\vg\cdot\ig < 0$ (cell delivering to grid):
                    overcharged cells get \emph{non-clamped}
                    ($v_{nc}^{*}$); undercharged cells get
                    \emph{clamped} ($v_c^{*}$).
              \item If $\vg\cdot\ig > 0$ (grid delivering to cell):
                    the assignment flips.
          \end{itemize}
          The flip ensures that regardless of power flow direction,
          the overcharged cell always ends up drawing more loading
          power and the undercharged cell draws less.
\end{enumerate}
\end{circuitbox}

The full worked case study (two cells, four time regions) and the
generalisation to $n$ cells (five possible positions of $V_{dc,H}^{*}$
in a four-cell sorted list) are treated in the next section.

\clearpage

\end{document}
