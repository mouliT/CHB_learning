% ============================================================
%  DC-Link Voltage Balancing Modulation for Cascaded H-Bridge Converters
%  Paper 3 Reference Notes
%  Source: [Vasishta et al.], IEEE Access
%  Figures stored in: figures/
%  Naming convention: P3_Fig[N]_[description].png
% ============================================================
\documentclass[12pt, a4paper]{article}

% ---- core packages ----
\usepackage[utf8]{inputenc}
\usepackage[T1]{fontenc}
\usepackage{amsmath, amssymb, amsthm}
\usepackage{bm}
\usepackage{siunitx}

% ---- figures ----
\usepackage{graphicx}
\graphicspath{{figures/}}           % all PNGs live here
\usepackage{float}
\usepackage{caption}
\usepackage{subcaption}

% ---- layout ----
\usepackage[a4paper, top=2.5cm, bottom=2.5cm, left=3cm, right=2.8cm]{geometry}
\usepackage{parskip}
\setlength{\parskip}{6pt}
\setlength{\parindent}{0pt}

% ---- hyperlinks ----
\usepackage[colorlinks=true, linkcolor=blue!60!black, citecolor=blue!50!black]{hyperref}
\usepackage{bookmark}
\makeatletter
\pdfstringdefDisableCommands{%
    \def\vspace#1{}%
    \def\rule#1#2{}%
    \def\color#1{}%
    \let\leavevmode@ifvmode\relax%
    \def\,{ }%
}
\makeatother

% ---- styled boxes ----
\usepackage{tcolorbox}
\tcbuselibrary{skins, breakable}

% ---- tables ----
\usepackage{booktabs}
\usepackage{array}
\usepackage{longtable}

% ---- section formatting ----
\usepackage{titlesec}
\usepackage{xcolor}
\definecolor{chbblue}{RGB}{10, 55, 120}
\definecolor{chborange}{RGB}{190, 70, 5}
\definecolor{chbgreen}{RGB}{8, 95, 38}
\definecolor{chbred}{RGB}{170, 25, 25}
\definecolor{lightblue}{RGB}{216, 232, 252}
\definecolor{lightorange}{RGB}{255, 244, 220}
\definecolor{lightgreen}{RGB}{216, 247, 228}
\definecolor{lightred}{RGB}{255, 228, 228}
\definecolor{lightgray}{RGB}{245, 245, 245}

\titleformat{\section}{\large\bfseries\color{chbblue}}{{\thesection.}}{0.5em}{}
            [\vspace{-4pt}\rule{\textwidth}{0.5pt}]
\titleformat{\subsection}{\normalsize\bfseries\color{chbblue!80!black}}{{\thesubsection}}{0.5em}{}

% ---- tcolorbox styles ----
\newtcolorbox{circuitbox}[1][Circuit Description]{
    colback=lightblue, colframe=chbblue, coltitle=white,
    fonttitle=\bfseries\small, title={#1},
    breakable, enhanced, left=4pt, right=4pt, top=4pt, bottom=4pt}

\newtcolorbox{statebox}[1][State Variables \& Parameters]{
    colback=lightorange, colframe=chborange, coltitle=white,
    fonttitle=\bfseries\small, title={#1},
    breakable, enhanced, left=4pt, right=4pt, top=4pt, bottom=4pt}

\newtcolorbox{eqnbox}[1][Governing Equations]{
    colback=lightgreen, colframe=chbgreen, coltitle=white,
    fonttitle=\bfseries\small, title={#1},
    breakable, enhanced, left=4pt, right=4pt, top=4pt, bottom=4pt}

\newtcolorbox{physbox}[1][Physical Interpretation]{
    colback=lightgray, colframe=gray!60, coltitle=white,
    fonttitle=\bfseries\small, title={#1},
    breakable, enhanced, left=4pt, right=4pt, top=4pt, bottom=4pt}

\newtcolorbox{warningbox}[1][Watch Out]{
    colback=lightred, colframe=chbred, coltitle=white,
    fonttitle=\bfseries\small, title={#1},
    breakable, enhanced, left=4pt, right=4pt, top=4pt, bottom=4pt}

% ---- custom commands ----
% Grid quantities
\newcommand{\vg}{v_{g}}                    % grid voltage (instantaneous)
\newcommand{\Vg}{V_{g}}                    % grid voltage (amplitude)
\newcommand{\ig}{i_{g}}                    % grid current (instantaneous)
\newcommand{\Ig}{I_{g}}                    % grid current (amplitude)
\newcommand{\omg}{\omega}                  % fundamental angular frequency
\newcommand{\phig}{\phi_{g}}               % power factor angle (grid)
% DC-link quantities
\newcommand{\Vdc}{V_{dc}}                  % nominal DC-link voltage (per cell)
\newcommand{\Vdci}[1]{V_{dc,#1}}          % DC-link voltage of cell i
\newcommand{\Vtot}{V_{dc,\mathrm{tot}}}    % total DC-link voltage (sum over all cells)
% Modulation
\newcommand{\Mi}{M_{i}}                    % modulation index of cell i
\newcommand{\phic}{\varphi}               % clamping angle
\newcommand{\vref}{v_{\mathrm{ref}}}       % modulation reference signal
% Loading power
\newcommand{\PH}{P_{H}}                    % loading power (generic)
\newcommand{\PHi}[1]{P_{H,#1}}            % loading power of cell i
\newcommand{\PHnc}{P_{H}^{\mathrm{nc}}}   % loading power, non-clamped mode
\newcommand{\PHcl}{P_{H}^{\mathrm{cl}}}   % loading power, clamped mode
% Miscellaneous
\newcommand{\figref}[1]{Figure~\ref{#1}}

% ---- figure entry macro ----
% Usage: \figentry{label}{filename}{caption}{width}
\newcommand{\figentry}[4]{%
    \begin{figure}[H]
        \centering
        \includegraphics[width=#4\textwidth]{#2}
        \caption{#3}
        \label{#1}
    \end{figure}%
}

% ============================================================
\title{\textbf{DC-Link Voltage Balancing Modulation}\\[4pt]
       \textbf{for Cascaded H-Bridge Converters}\\[6pt]
       \large Paper 3 --- Circuit Topology, Control Architecture,\\
       \large Loading Power Theory, and Sorting Algorithm\\[4pt]
       \normalsize IEEE Access}
\date{\today}
% ============================================================

\begin{document}
\maketitle
\tableofcontents
\clearpage

% ============================================================
\section{How to Use This Document}
% ============================================================

Each section follows this standard structure:

\begin{enumerate}
    \item \textbf{Figure} --- the relevant figure from the paper
    \item \textbf{Circuit Description} (blue box) --- what components are present
          and how they are connected; what the topology is doing
    \item \textbf{State Variables \& Parameters} (orange box) --- every symbol
          that appears in the figure, with units and physical meaning
    \item \textbf{Governing Equations} (green box) --- the equations that
          describe this element of the system, derived from first principles
    \item \textbf{Physical Interpretation} (gray box) --- what the result
          \emph{means}: energy flow, design trade-offs, why this matters
\end{enumerate}

All figures are from the source paper on DC-link voltage balancing modulation
for Cascaded H-Bridge converters.

\medskip
\textbf{Vector figures (Figs.\,1--7)} are circuit diagrams and block diagrams
drawn as PDF vector graphics. They are shown here as full-page renders
(\texttt{P3\_pageXX\_render.png}). Crop the relevant figure from the render
and rename following the \texttt{P3\_Fig[N]\_description.png} convention before
final use.

\clearpage

% ============================================================
\section{The Problem: Why Does the DC-Link Become Unbalanced?}
\label{sec:problem}
% ============================================================

\subsection{Start Here: What Is a Cascaded H-Bridge?}

Before the equations, the physical picture.

A \textbf{Cascaded H-Bridge (CHB)} inverter is built by stacking multiple
H-bridge converter cells in series. Each cell has:
\begin{itemize}
    \item its own isolated DC-link capacitor ($C_{dc,i}$, voltage $\Vdci{i}$),
    \item four switches forming a full H-bridge,
    \item a separate DC source (solar panel, battery, or simply a capacitor).
\end{itemize}
The output voltages of all cells add up in series. If there are $n$ cells,
the total AC output is:
\[
    \vg = \sum_{i=1}^{n} v_{i}(t)
\]
where $v_i(t)$ is the output voltage of cell $i$.

\medskip
\textbf{Why use CHB?} High AC voltage (e.g., medium voltage grid) from
low-voltage DC sources (e.g., 200\,V solar panels) without a transformer.
Each cell contributes a small voltage step. The staircase sum approaches a
sinusoidal waveform with low harmonic distortion.

\medskip
\textbf{The problem:} Real DC sources are not identical. Solar panels on
different roof faces receive different irradiance. Batteries in different
strings age differently. Each cell's DC source delivers a different power
$P_{i}$. This is intentional and useful --- but it forces the DC-link
voltages $\Vdci{i}$ of the cells to diverge over time if nothing is done
about it.

\medskip
\textbf{Why does diverging $\Vdci{i}$ matter?}
\begin{itemize}
    \item The total output voltage can no longer be shared equally: some
          cells produce more volts per switch state than others. The output
          waveform becomes distorted.
    \item Capacitor over-voltage: if $\Vdci{i}$ rises too high, the capacitor
          or switches are damaged.
    \item Under-voltage: if $\Vdci{i}$ drops, the cell cannot contribute its
          share of the AC voltage. Output voltage sags.
\end{itemize}

The task of \textbf{DC-link voltage balancing} is to keep all $\Vdci{i}$
equal to a common reference $\Vdc$ despite the cells having different source
powers.

\figentry{fig:P3Fig1}{P3_page01_render.png}%
{Full page 1 render. \textbf{Fig.\,1 (CHB topology)} appears in the
lower-right area of this page. It shows $n$ H-bridge cells in series, each
with an isolated DC-link capacitor and unequal source powers
$P_{1H} \neq P_{2H} \neq \cdots \neq P_{nH}$. Crop this render to extract
Fig.\,1 and rename to \texttt{P3\_Fig1\_CHB\_topology.png}.}{0.6}

\begin{circuitbox}[Circuit Description --- CHB Topology (Fig.\,1)]
The CHB string contains $n$ identical H-bridge cells connected in series at
their AC output terminals. Inside each cell $i$:
\begin{itemize}
    \item \textbf{DC-link capacitor} $C_{dc,i}$: stores the DC energy.
          Voltage $\Vdci{i}$ can differ from cell to cell.
    \item \textbf{H-bridge switches} ($S_{i,1}$--$S_{i,4}$, typically IGBTs):
          chop the DC voltage into AC via pulse-width modulation.
    \item \textbf{DC source} (solar panel / battery / diode rectifier):
          delivers power $P_{iH}$ into the DC-link. The subscript $H$
          denotes the H-bridge loading power (defined precisely in
          Section~\ref{sec:loading_power}).
\end{itemize}
The AC output of all cells is summed in series. The sum drives the grid
current $\ig$ through the series-connected filter inductor $L_{f}$.

\medskip
Key structural point: the cells are \textbf{AC-coupled} (series output) but
\textbf{DC-isolated} (each has its own capacitor). This means the AC current
$\ig$ is the same for every cell, but each cell's DC voltage $\Vdci{i}$ is
independent.
\end{circuitbox}

\begin{statebox}
\renewcommand{\arraystretch}{1.3}
\begin{tabular}{p{0.12\linewidth} p{0.32\linewidth} p{0.46\linewidth}}
\toprule
\textbf{Symbol} & \textbf{Quantity} & \textbf{Role / Typical Value} \\
\midrule
$n$        & Number of H-bridge cells  & 3 in prototype (230\,V$_{\rm rms}$ grid, 180\,V per cell) \\
$\Vdci{i}$ & DC-link voltage, cell $i$ & Target: 180\,V each; varies if unbalanced \\
$\Vtot$    & $\sum_{i=1}^{n}\Vdci{i}$  & Total DC-link sum; controlled by outer loop \\
$C_{dc,i}$ & DC-link capacitance        & Sets voltage ripple magnitude \\
$\PHi{i}$  & Loading power, cell $i$    & Net power from DC source into $C_{dc,i}$; positive = charging \\
$\ig$      & Grid current               & Same through all cells; controlled by inner loop \\
$L_{f}$    & Filter inductance          & Grid-side series inductor \\
\bottomrule
\end{tabular}
\end{statebox}

\begin{physbox}[Physical Interpretation --- Why Voltages Diverge]
At steady state, each cell's DC-link capacitor voltage is constant, which
means the power flowing into the capacitor equals the power flowing out.
The power in comes from the DC source ($P_{i,\mathrm{source}}$). The power
out is delivered to the AC grid via the H-bridge.

If cell 1's source delivers 500\,W and cell 2's source delivers 200\,W,
but each cell must deliver the same fraction of the total AC power (because
the AC current $\ig$ is the same through both), there is a mismatch:
cell 1 is over-sourced and cell 2 is under-sourced. Without balancing, this
mismatch charges $C_{dc,1}$ and discharges $C_{dc,2}$ indefinitely.

\medskip
The balancing controller must redistribute how much AC power each cell
contributes, so that each cell's capacitor reaches a steady-state equilibrium
at $\Vdc$.
\end{physbox}

\clearpage
% ============================================================
\section{The Control Architecture}
\label{sec:control}
% ============================================================

\figentry{fig:P3Fig2}{P3_page02_render.png}%
{Full page 2 render. \textbf{Fig.\,2 (Control scheme)} appears on this page.
It shows the three-loop control: outer $\Vtot$ loop, inner $\ig$ loop, and
the balancing modulation block. Crop to \texttt{P3\_Fig2\_control\_scheme.png}.}{0.6}

The control has three layers, working from slow (outer) to fast (inner):

\subsection{Outer Loop: Total DC-Link Voltage Control}

\textbf{What it controls:} the sum $\Vtot = \sum_{i=1}^{n} \Vdci{i}$.

\textbf{How:} A PI controller compares $\Vtot$ to the reference $nV_{dc,\mathrm{ref}}$.
The output is the grid current reference amplitude $\Ig^{*}$. A larger
$\Ig^{*}$ means more current from the grid, which charges all cells
simultaneously.

\textbf{Timescale:} slow --- this loop responds to changes in total load
or total source power. It does not distinguish between cells.

\subsection{Inner Loop: Grid Current Control}

\textbf{What it controls:} $\ig(t)$, the actual grid current.

\textbf{How:} A fast current controller (PI, PR, or deadbeat) generates a
total modulation reference $\vref(t)$ that is shared and then split among
the $n$ cells.

\textbf{Timescale:} fast (typically one or a few switching periods).

\subsection{Balancing Loop: Individual Cell Voltage Control}

\textbf{What it controls:} each $\Vdci{i}$ individually, keeping them equal.

\textbf{How:} this is the key contribution of the paper. The balancing loop
assigns each cell a modulation \emph{strategy} (clamped or non-clamped) and
adjusts the individual cell modulation index $\Mi$ and clamping angle $\phic$.
By doing so, the \textbf{loading power} $\PHi{i}$ of each cell is set to
exactly the value needed to charge or discharge $C_{dc,i}$ to the target.

\textbf{Timescale:} can be set very fast --- the loading power can be changed
within one or two fundamental cycles (10--20\,ms at 50\,Hz), giving the
$\sim$140\,ms settling time demonstrated experimentally.

\begin{physbox}[Why Three Loops?]
The outer loop sees the total energy. It does not care which cell holds that
energy --- just that the sum is right. Think of it as controlling the total
water level in a set of connected tanks.

The balancing loop then redistributes energy between tanks without changing
the total. The key is that the redistribution is done through the modulation
--- by changing \emph{how each cell switches}, not by adding hardware.

The inner current loop ensures the redistribution does not distort the
grid current.
\end{physbox}

\clearpage
% ============================================================
\section{Conventional Balancing: The PI-Based Approach}
\label{sec:conventional}
% ============================================================

\figentry{fig:P3Fig3}{P3_page03_render.png}%
{Full page 3 render. \textbf{Fig.\,3 (Conventional PI balancing)} appears on
this page. It shows a PI controller for each cell correcting its modulation
index based on the voltage error $\Vdci{i} - \Vdc$. Also on this page:
\textbf{Fig.\,4 (sorting algorithm)} and \textbf{Fig.\,5 (two-cell case)}.
Crop individually and rename as:
\texttt{P3\_Fig3\_PI\_balancing.png},
\texttt{P3\_Fig4\_sorting\_algorithm.png},
\texttt{P3\_Fig5\_two\_cell\_case.png}.}{0.6}

\subsection{How the Conventional Method Works}

Each cell gets a PI controller that watches the voltage error
$\varepsilon_i = \Vdci{i} - V_{dc,\mathrm{ref}}$.

The PI output is a small modulation correction $\Delta M_i$:
\[
    \Delta M_i = \left(K_P + \frac{K_I}{s}\right) \varepsilon_i
\]

This correction is added to the base modulation reference (which all cells
share from the inner current loop):
\[
    M_i^{\mathrm{total}}(t) = M_{\mathrm{base}}(t) + \Delta M_i
\]

A cell with $\Vdci{i} > V_{dc,\mathrm{ref}}$ gets a larger modulation index
(delivers more AC power, consuming more of its stored DC energy, pulling the
voltage down). A cell with $\Vdci{i} < V_{dc,\mathrm{ref}}$ gets a smaller
modulation index (delivers less AC power, allowing the source to charge the
capacitor, pulling the voltage up).

\subsection{Why Is the Conventional Method Slow?}

\begin{warningbox}[Fundamental Limitation of PI Balancing]
The PI controller generates $\Delta M_i$, which modifies the \emph{average}
power delivery of each cell. But the relationship between modulation index
change $\Delta M_i$ and power change $\Delta P_i$ is indirect and
gain-limited: the correction is small compared to the rated modulation
depth (otherwise the grid current distorts). This small correction acts
against the large stored energy in $C_{dc,i}$.

Result: the PI must integrate slowly over many cycles to accumulate enough
energy transfer. Measured settling time: $\sim$\,1850\,ms.

The proposed method bypasses this limitation by \emph{directly setting} the
loading power of each cell --- a fundamentally faster degree of freedom.
\end{warningbox}

\clearpage
% ============================================================
\section{Key Concept: What Is Loading Power?}
\label{sec:loading_power}
% ============================================================

\subsection{The AC Power of a Single H-Bridge Cell}

Each cell $i$ produces an AC output voltage $v_i(t)$. The instantaneous
power delivered to the grid by cell $i$ is:
\[
    p_i(t) = v_i(t) \cdot \ig(t)
\]

The grid current is sinusoidal:
\[
    \ig(t) = \Ig \sin(\omg t)
\]

For a standard sinusoidal (non-clamped) modulation with index $\Mi$ and
DC-link voltage $\Vdci{i}$, the cell output voltage is:
\[
    v_i(t) = \Vdci{i} \cdot \Mi \cdot \sin(\omg t + \theta_i)
\]

where $\theta_i$ is the phase shift of cell $i$'s reference (all cells
share the same phase in single-phase operation: $\theta_i = 0$).

\subsection{The DC-Link Capacitor Power Balance}

The DC source of cell $i$ delivers power $P_{i,\mathrm{src}}$ into the DC-link.
The H-bridge extracts power $p_i(t)$ from the DC-link and delivers it to the
grid. The net power flowing into the capacitor is:
\[
    p_{C,i}(t) = P_{i,\mathrm{src}} - p_i(t)
\]

The \textbf{loading power} $\PH$ is defined as the \emph{average} net power
flow through the DC-link capacitor over one fundamental cycle. If we define
the average of $p_i(t)$ as $\langle p_i \rangle$, then:
\[
    \PH = P_{i,\mathrm{src}} - \langle p_i \rangle
\]

\begin{itemize}
    \item If $\PH > 0$: the capacitor is receiving net energy --- $\Vdci{i}$ \emph{rises}.
    \item If $\PH < 0$: the capacitor is losing net energy --- $\Vdci{i}$ \emph{falls}.
    \item If $\PH = 0$: the capacitor voltage is constant (balanced).
\end{itemize}

For balancing, we want to set $\PH$ for each cell to exactly whatever value
drives $\Vdci{i}$ back to $V_{dc,\mathrm{ref}}$ as fast as possible.

\begin{physbox}[Analogy: Loading Power as a Water Tap]
Think of the DC-link capacitor as a water tank. The DC source is an inlet
pipe (fills the tank at rate $P_{i,\mathrm{src}}$). The H-bridge is a drain
(empties the tank at rate $p_i(t)$).

Loading power $\PH$ = inlet rate minus drain rate = net change in tank level.

To raise the voltage (charge the tank): set $\PH > 0$, i.e., make the drain
smaller than the inlet. This means making the modulation index smaller so the
H-bridge extracts less power.

To lower the voltage (drain the tank): set $\PH < 0$, i.e., make the drain
larger than the inlet. This means increasing the modulation contribution so the
H-bridge extracts more power from the DC-link.

The key insight of this paper: \emph{the modulation strategy (clamped vs.
non-clamped) directly sets the average drain rate}, so by choosing the
modulation, we set $\PH$ directly and immediately --- without waiting for a
slow PI to integrate.
\end{physbox}

\clearpage
% ============================================================
\section{Clamped vs.\ Non-Clamped Modulation}
\label{sec:clamped}
% ============================================================

\figentry{fig:P3Fig7}{P3_page04_render.png}%
{Full page 4 render. \textbf{Fig.\,7 (Clamped modulation signal)} appears
on this page. It shows the modulation reference waveform with the clamped
(flat-top) regions defined by the clamping angle $\phic$.
Crop to \texttt{P3\_Fig7\_clamped\_modulation\_signal.png}.
Also visible on this page: Fig.\,8 (3D loading power surfaces) and Fig.\,9.}{0.6}

\subsection{Non-Clamped (Standard Sinusoidal) Modulation}

In \textbf{non-clamped} modulation, the reference signal for cell $i$ is a
pure sinusoid:
\[
    \vref^{\mathrm{nc}}(t) = \Mi \cdot \sin(\omg t)
\]

The modulation index $\Mi \leq 1$ at all times (never hits the rail).
The H-bridge switches continuously throughout the cycle --- the duty cycle
varies smoothly.

The average (active) power delivered to the grid is:
\[
    \langle p_i \rangle^{\mathrm{nc}}
    = \frac{\Vdci{i} \cdot \Mi \cdot \Ig}{2} \cos(\phig)
\]

The loading power for non-clamped operation is a function of $\Mi$, $\Ig$,
and the power factor angle $\phig$.

\subsection{Clamped Modulation}

In \textbf{clamped} modulation, the reference signal is clamped to $\pm 1$
during portions of the cycle where it would otherwise be near the rails.
The clamped region is controlled by the \textbf{clamping angle} $\phic$:

\[
    \vref^{\mathrm{cl}}(t) =
    \begin{cases}
        +1 & \text{if } \Mi\sin(\omg t) > \cos(\phic) \\
        -1 & \text{if } \Mi\sin(\omg t) < -\cos(\phic) \\
        \Mi\sin(\omg t) & \text{otherwise}
    \end{cases}
\]

When the reference is clamped to $+1$: the upper switch of the H-bridge is
fully on (no switching). When clamped to $-1$: the lower switch is fully on.
During clamped intervals, there are \emph{no switching losses} and the cell
output is simply $\pm \Vdci{i}$.

\begin{statebox}
\renewcommand{\arraystretch}{1.3}
\begin{tabular}{p{0.13\linewidth} p{0.35\linewidth} p{0.44\linewidth}}
\toprule
\textbf{Symbol} & \textbf{Quantity} & \textbf{Notes} \\
\midrule
$\Mi$    & Modulation index, cell $i$  & $0 \leq \Mi \leq 1$; sets fundamental output amplitude \\
$\phic$  & Clamping angle              & $0 \leq \phic \leq \pi/2$; controls how much of cycle is clamped \\
         &                             & $\phic = 0$ $\Rightarrow$ no clamping (pure sinusoidal) \\
         &                             & $\phic = \pi/2$ $\Rightarrow$ fully clamped (always at $\pm 1$) \\
$\phig$  & Power factor angle          & Angle between $\vg$ and $\ig$; fixed by load \\
$\PHnc$  & Loading power, non-clamped  & Function of $\Mi$, $\Ig$, $\phig$ only \\
$\PHcl$  & Loading power, clamped      & Function of $\Mi$, $\phic$, $\Ig$, $\phig$ \\
\bottomrule
\end{tabular}
\end{statebox}

\begin{physbox}[Why Does Clamping Change the Loading Power?]
When the reference is clamped, the H-bridge output voltage during the clamped
interval is exactly $\pm \Vdci{i}$ (a DC level), not a PWM-averaged fraction.
The power extracted from the DC-link during the clamped interval is
$\pm \Vdci{i} \cdot \ig(t)$ instantaneously.

By varying $\phic$, the fraction of the cycle that is clamped changes.
This changes the total energy extracted per cycle --- i.e., the loading power
$\PH$.

The clamped and non-clamped modes therefore give different loading powers for
the same grid current, even with the same fundamental amplitude. This
\emph{extra degree of freedom} (clamped vs.\ non-clamped, and the choice of
$\phic$) is what the proposed algorithm exploits for fast balancing.
\end{physbox}

\clearpage
% ============================================================
\section{The Proposed Sorting Algorithm}
\label{sec:algorithm}
% ============================================================

\subsection{The Core Idea}

The proposed method decides, at the start of each fundamental cycle:
\begin{enumerate}
    \item Which cells need their DC-link \emph{charged} (voltage too low)?
    \item Which cells need their DC-link \emph{discharged} (voltage too high)?
    \item For each cell, what modulation strategy (clamped or non-clamped)?
    \item What values of $\Mi$ and $\phic$?
\end{enumerate}

The answer comes from \textbf{sorting by required loading power} and
assigning modes accordingly.

\figentry{fig:P3Fig4}{P3_page03_render.png}%
{Full page 3 render (same as \figref{fig:P3Fig3}). \textbf{Fig.\,4 (Sorting
algorithm flowchart)} is on this page. Crop to
\texttt{P3\_Fig4\_sorting\_algorithm.png}.}{0.6}

\subsection{Algorithm Steps}

\textbf{Step 1 --- Compute voltage errors.}
For each cell $i$, compute the error from the reference:
\[
    \varepsilon_i = V_{dc,\mathrm{ref}} - \Vdci{i}
\]
A positive error means the cell is under-voltage (needs charging, $\PH > 0$).
A negative error means the cell is over-voltage (needs discharging, $\PH < 0$).

\medskip
\textbf{Step 2 --- Rank cells by error magnitude.}
Sort the cells by $|\varepsilon_i|$ in descending order. The cell furthest
from the target gets the most aggressive loading power correction.

\medskip
\textbf{Step 3 --- Assign modulation mode.}
\begin{itemize}
    \item Cells needing discharge ($\varepsilon_i < 0$): assign \textbf{clamped}
          modulation with a large $\phic$ (high loading power drain).
    \item Cells needing charge ($\varepsilon_i > 0$): assign \textbf{non-clamped}
          modulation to maximize loading power input.
\end{itemize}

\medskip
\textbf{Step 4 --- Compute $\Mi$ and $\phic$.}
Using the loading power expressions (Section~\ref{sec:loading_power_derivation}),
find the $\Mi$ and $\phic$ that give the exact $\PH$ required to drive
$\varepsilon_i$ to zero in minimum time.

\medskip
\textbf{Step 5 --- Verify constraint: total voltage sum.}
The per-cell modulation references must still produce the correct grid
voltage. The outer loop sets the constraint:
\[
    \sum_{i=1}^{n} v_i^{\mathrm{fund}}(t) = \Vtot \cdot M_{\mathrm{base}} \cdot \sin(\omg t)
\]
Adjust $\Mi$ values proportionally if needed to satisfy this constraint.

\begin{physbox}[Why Is This Faster Than PI?]
The PI method changes $\Delta M_i$ by a small increment each cycle, limited
by stability. The integral accumulates over many cycles before the voltage
is corrected.

The sorting algorithm directly \emph{sets} the loading power target. Instead
of nudging $M_i$ by $\Delta M_i$ and waiting for the energy to integrate, the
algorithm computes the exact modulation parameters that deliver the required
$\PH$ in one step.

This is the difference between slowly adjusting a tap (PI) vs.\ opening a
valve all the way and calculating exactly when to close it (sorting algorithm).

Measured result: 140\,ms settling time vs.\ 1850\,ms for the PI approach
--- a factor of 13$\times$ faster.
\end{physbox}

\clearpage
% ============================================================
\section{Two-Cell Case Study}
\label{sec:two_cell}
% ============================================================

\figentry{fig:P3Fig5}{P3_page03_render.png}%
{Full page 3 render. \textbf{Fig.\,5 (Two-cell case)} is on this page.
It shows the clamped and non-clamped regions in the $(M, \phic)$ operating
plane for the two-cell case. Crop to \texttt{P3\_Fig5\_two\_cell\_case.png}.}{0.6}

With $n = 2$ cells, the algorithm is simplest to understand.

\subsection{Setup}

Two cells: cell\,1 and cell\,2. Both contribute to the same grid current.
The total modulation must satisfy:
\[
    M_1 \sin(\omg t) + M_2 \sin(\omg t) = M_{\mathrm{total}} \sin(\omg t)
\]
so $M_1 + M_2 = M_{\mathrm{total}}$ (in the balanced case).

Suppose cell\,1 is over-voltage ($\Vdci{1} > V_{dc,\mathrm{ref}}$) and
cell\,2 is under-voltage ($\Vdci{2} < V_{dc,\mathrm{ref}}$).

\subsection{Assignment}

The algorithm assigns:
\begin{itemize}
    \item \textbf{Cell\,1} (over-voltage, needs discharge): \textbf{clamped mode},
          with $\phic > 0$. The clamped mode increases the loading power drain
          from cell\,1's DC-link, causing $\Vdci{1}$ to fall.
    \item \textbf{Cell\,2} (under-voltage, needs charge): \textbf{non-clamped mode},
          maximizing the loading power input into cell\,2's DC-link, causing
          $\Vdci{2}$ to rise.
\end{itemize}

The modulation indices $M_1$, $M_2$, and the clamping angle $\phic$ of
cell\,1 are computed from the loading power expressions so that both cells
reach $V_{dc,\mathrm{ref}}$ simultaneously.

\subsection{Clamped vs.\ Non-Clamped Regions in the Operating Plane}

The figure (Fig.\,5) shows which combinations of $M$ and $\phic$ are feasible
for the clamped and non-clamped cells. Not all $(M, \phic)$ points are
reachable simultaneously --- the two cells share the total modulation budget
$M_{\mathrm{total}}$. The algorithm searches within the feasible region for
the optimal point.

\begin{physbox}[What the Two-Cell Case Shows]
The two-cell case makes it clear that the clamped/non-clamped assignment is
not arbitrary. Assigning the over-voltage cell to clamped mode (not
non-clamped) is deliberate: clamped mode gives higher loading power drain
per unit modulation index, which is exactly what is needed to bring an
over-voltage cell down quickly.

Assigning the under-voltage cell to non-clamped mode maximizes the loading
power input (charging the DC-link). The non-clamped cell absorbs energy from
the grid current during all parts of the cycle --- not just the clamped
intervals.
\end{physbox}

\clearpage
% ============================================================
\section{Four-Cell Extension and Five Operating Cases}
\label{sec:four_cell}
% ============================================================

\figentry{fig:P3Fig6}{P3_page04_render.png}%
{Full page 4 render. \textbf{Fig.\,6 (Five cases for four cells)} may appear
on this page. It shows the five possible assignments of clamped and non-clamped
modes when four cells are present, depending on how many cells need charge vs.\
discharge. Crop to \texttt{P3\_Fig6\_five\_cases\_four\_cells.png}.}{0.6}

With $n = 4$ cells, the number of cells needing discharge vs.\ charge can
range from 0:4 to 4:0. The paper identifies \textbf{five distinct cases}
based on this split:

\begin{center}
\renewcommand{\arraystretch}{1.3}
\begin{tabular}{clll}
\toprule
\textbf{Case} & \textbf{Cells to discharge} & \textbf{Cells to charge} & \textbf{Mode assignment} \\
\midrule
I   & 0 & 4 & All non-clamped \\
II  & 1 & 3 & 1 clamped + 3 non-clamped \\
III & 2 & 2 & 2 clamped + 2 non-clamped \\
IV  & 3 & 1 & 3 clamped + 1 non-clamped \\
V   & 4 & 0 & All clamped \\
\bottomrule
\end{tabular}
\end{center}

For each case, the loading power constraints produce a set of equations for
the unknown modulation indices and clamping angles. The sorting algorithm
solves these equations using the closed-form loading power expressions.

\begin{physbox}[Symmetry of the Five Cases]
Cases I and V are symmetric: all cells either need charge (Case I) or all
need discharge (Case V). In these cases all cells get the same mode, and the
balancing is symmetric.

Cases II--IV are asymmetric: some cells get clamped mode, others non-clamped.
The loading power difference between the two modes drives the inter-cell
energy redistribution.

In practice, Cases II--IV are the most common during transients, since
balanced starting conditions rarely have all cells on the same side of the
target.
\end{physbox}

\clearpage
% ============================================================
\section{Loading Power Derivation}
\label{sec:loading_power_derivation}
% ============================================================

\subsection{Non-Clamped Loading Power $\PHnc$}

For a cell in non-clamped mode, the output voltage is:
\[
    v_i(t) = \Vdci{i} \cdot \Mi \cdot \sin(\omg t)
\]

The instantaneous power from the DC-link to the grid:
\[
    p_i(t) = \Vdci{i} \cdot \Mi \cdot \sin(\omg t) \cdot \Ig \sin(\omg t - \phig)
\]

Expanding using the product-to-sum identity:
\[
    p_i(t) = \frac{\Vdci{i} \cdot \Mi \cdot \Ig}{2}
             \left[\cos(\phig) - \cos(2\omg t - \phig)\right]
\]

The first term is the average (DC) power --- real power to the grid.
The second term is a double-frequency ripple that alternately charges and
discharges $C_{dc,i}$. The \textbf{average} is:
\[
    \langle p_i \rangle^{\mathrm{nc}} = \frac{\Vdci{i} \cdot \Mi \cdot \Ig}{2} \cos(\phig)
\]

The loading power (net power into the capacitor) is:
\[
    \boxed{\PHnc = P_{i,\mathrm{src}} - \frac{\Vdci{i} \cdot \Mi \cdot \Ig}{2} \cos(\phig)}
\]

To \emph{increase} $\PHnc$ (charge the capacitor): decrease $\Mi$. Less power
extracted from the DC-link means more stays in the capacitor.

\subsection{Clamped Loading Power $\PHcl$}

For clamped modulation, the output voltage is no longer a pure sinusoid.
During the clamped intervals (when the reference hits $\pm 1$), $v_i = \pm\Vdci{i}$.

The average power is computed by integrating over one full cycle, treating
the sinusoidal and clamped intervals separately. The result (from Fourier
analysis of the clamped waveform) is:

\[
    \langle p_i \rangle^{\mathrm{cl}}
    = \frac{\Vdci{i} \cdot \Ig}{2\pi}
      \left[ \Mi \pi \cos(\phig) + 2 \cos(\phic)\sin(\phig) \right]
    \quad \text{[single-phase, unity fundamental]}
\]

\begin{warningbox}[Note: Verify Against Paper]
The exact coefficient form above should be verified against the paper's
equation for $P_H^{cl}$. The key structure is correct: $\PHcl$ depends on
\emph{both} $\Mi$ and $\phic$, giving the algorithm two degrees of freedom
for the clamped cells. The non-clamped cells have only $\Mi$ as a free
variable (one degree of freedom).
\end{warningbox}

\begin{eqnbox}[Summary: Loading Power Expressions]
\textbf{Non-clamped cell} (one free variable: $\Mi$):
\[
    \PHnc(\Mi)
    = P_{i,\mathrm{src}} - \frac{\Vdci{i} \cdot \Mi \cdot \Ig}{2} \cos(\phig)
\]

\textbf{Clamped cell} (two free variables: $\Mi$, $\phic$):
\[
    \PHcl(\Mi, \phic)
    = P_{i,\mathrm{src}} - \frac{\Vdci{i} \cdot \Ig}{2\pi}
      \left[ \Mi \pi \cos(\phig) + 2\cos(\phic)\sin(\phig) \right]
\]

\textbf{Constraint} (total grid voltage):
\[
    \sum_{i=1}^{n} \Vdci{i} \cdot \Mi \cdot \sin(\omg t + \theta_i)
    = \Vtot \cdot M_{\mathrm{ref}} \cdot \sin(\omg t)
\]

The sorting algorithm solves these equations simultaneously for all cells
to find the optimal $\{\Mi, \phic\}$ set.
\end{eqnbox}

\clearpage
% ============================================================
\section{Loading Power Surface: 3D Visualization (Fig.\,8)}
\label{sec:3d_surface}
% ============================================================

\begin{figure}[H]
    \centering
    \begin{subfigure}[b]{0.48\textwidth}
        \centering
        \includegraphics[width=\textwidth]{P3_Fig8a_3D_clamped_loading_power.png}
        \caption{Clamped mode loading power $\PHcl$ as a function of
                 modulation index $M$ and clamping angle $\phic$.}
        \label{fig:P3Fig8a}
    \end{subfigure}
    \hfill
    \begin{subfigure}[b]{0.48\textwidth}
        \centering
        \includegraphics[width=\textwidth]{P3_Fig8b_3D_nonclamped_loading_power.png}
        \caption{Non-clamped mode loading power $\PHnc$ as a function of
                 modulation index $M$.}
        \label{fig:P3Fig8b}
    \end{subfigure}
    \caption{Fig.\,8 --- Loading power surfaces for clamped and non-clamped modes
             [verify figure identification against paper].}
    \label{fig:P3Fig8}
\end{figure}

\begin{figure}[H]
    \centering
    \begin{subfigure}[b]{0.48\textwidth}
        \centering
        \includegraphics[width=\textwidth]{P3_Fig8c_loading_power_strip.png}
        \caption{Companion panel (c).}
    \end{subfigure}
    \hfill
    \begin{subfigure}[b]{0.48\textwidth}
        \centering
        \includegraphics[width=\textwidth]{P3_Fig8d_loading_power_strip.png}
        \caption{Companion panel (d).}
    \end{subfigure}
    \\[6pt]
    \begin{subfigure}[b]{0.48\textwidth}
        \centering
        \includegraphics[width=\textwidth]{P3_Fig8e_loading_power_strip.png}
        \caption{Companion panel (e).}
    \end{subfigure}
    \hfill
    \begin{subfigure}[b]{0.48\textwidth}
        \centering
        \includegraphics[width=\textwidth]{P3_Fig8f_loading_power_strip.png}
        \caption{Companion panel (f).}
    \end{subfigure}
    \caption{Fig.\,8 additional panels --- verify these against paper.
             They may show 2D projections of the loading power surface
             at fixed $M$ or $\phic$ values, or may belong to Fig.\,7
             (clamped modulation waveforms).}
    \label{fig:P3Fig8strips}
\end{figure}

\begin{physbox}[Reading the 3D Surface]
The 3D surface plots show:
\begin{itemize}
    \item \textbf{x-axis}: modulation index $M$ (0 to 1)
    \item \textbf{y-axis}: clamping angle $\phic$ (0 to $\pi/2$, clamped mode only)
    \item \textbf{z-axis}: loading power $\PH$ (in Watts, normalized, or per-unit)
\end{itemize}

For the \textbf{clamped surface}: at $\phic = 0$ (no clamping) the surface
matches the non-clamped value. As $\phic$ increases, the loading power changes
because the output voltage waveform shape changes. The surface shows the
full range of achievable $\PH$ values by varying both $M$ and $\phic$.

For the \textbf{non-clamped surface}: it is a 2D curve (function of $M$ only),
shown as a flat ribbon on the 3D axes.

The algorithm uses these surfaces to look up the $(M, \phic)$ that achieves
the required $\PH$ for each cell.
\end{physbox}

\clearpage
% ============================================================
\section{Performance Comparison: Maximum Loading Power (Fig.\,9)}
\label{sec:performance_comparison}
% ============================================================

\figentry{fig:P3Fig9}{P3_Fig9_max_loading_power_comparison.png}%
{Fig.\,9 --- Maximum achievable loading power as a function of modulation
 index $M$, comparing the proposed sorting algorithm (solid) to the
 conventional PI-based method (dashed). The proposed method achieves higher
 maximum loading power across the full range of $M$, enabling faster
 DC-link voltage balancing.}{0.85}

\begin{physbox}[What Fig.\,9 Shows and Why It Matters]
Fig.\,9 plots the maximum loading power $|\PH|_{\max}$ achievable at each
modulation index $M$, for both the conventional and proposed methods.

A higher maximum loading power means:
\begin{itemize}
    \item More energy can be transferred per cycle.
    \item A larger voltage error can be corrected in the same amount of time.
    \item Or equivalently: the same voltage error is corrected in less time.
\end{itemize}

The proposed algorithm is better because:
\begin{enumerate}
    \item It uses \emph{both} $M$ and $\phic$ as free variables (clamped cells),
          giving more combinations that achieve high $|\PH|$.
    \item The sorting step ensures the most favorable mode (clamped vs.\
          non-clamped) is always assigned to each cell.
    \item The conventional PI method is limited to small $\Delta M$ corrections
          (otherwise it distorts the current), capping the achievable loading power.
\end{enumerate}
\end{physbox}

\clearpage
% ============================================================
\section{Simulation Results (Figs.\,10 and 11)}
\label{sec:simulation}
% ============================================================

\subsection{Fig.\,10 --- Simulation: Conventional vs.\ Proposed Control}

\begin{figure}[H]
    \centering
    \begin{subfigure}[b]{0.48\textwidth}
        \centering
        \includegraphics[width=\textwidth]{P3_Fig10a_sim_conv_vdc.png}
        \caption{Conventional: DC-link voltages $\Vdci{i}$.}
        \label{fig:P3Fig10a}
    \end{subfigure}
    \hfill
    \begin{subfigure}[b]{0.48\textwidth}
        \centering
        \includegraphics[width=\textwidth]{P3_Fig10d_sim_prop_vdc.png}
        \caption{Proposed: DC-link voltages $\Vdci{i}$.}
        \label{fig:P3Fig10d}
    \end{subfigure}
    \\[4pt]
    \begin{subfigure}[b]{0.48\textwidth}
        \centering
        \includegraphics[width=\textwidth]{P3_Fig10b_sim_conv_Pload.png}
        \caption{Conventional: loading powers $\PHi{i}$.}
        \label{fig:P3Fig10b}
    \end{subfigure}
    \hfill
    \begin{subfigure}[b]{0.48\textwidth}
        \centering
        \includegraphics[width=\textwidth]{P3_Fig10e_sim_prop_Pload.png}
        \caption{Proposed: loading powers $\PHi{i}$.}
        \label{fig:P3Fig10e}
    \end{subfigure}
    \\[4pt]
    \begin{subfigure}[b]{0.48\textwidth}
        \centering
        \includegraphics[width=\textwidth]{P3_Fig10c_sim_conv_modindex.png}
        \caption{Conventional: modulation indices $\Mi$.}
        \label{fig:P3Fig10c}
    \end{subfigure}
    \hfill
    \begin{subfigure}[b]{0.48\textwidth}
        \centering
        \includegraphics[width=\textwidth]{P3_Fig10f_sim_prop_modindex.png}
        \caption{Proposed: modulation indices $\Mi$.}
        \label{fig:P3Fig10f}
    \end{subfigure}
    \caption{Fig.\,10 --- Simulation comparison. Left column: conventional
             PI balancing. Right column: proposed sorting algorithm. Top row:
             DC-link voltages. Middle row: loading powers. Bottom row:
             modulation indices.}
    \label{fig:P3Fig10}
\end{figure}

\begin{physbox}[How to Read Fig.\,10]
\textbf{Row 1 (DC-link voltages)}: shows how fast the voltages converge to
the common reference. The proposed method converges in $\sim$140\,ms; the
conventional method in $\sim$1850\,ms (see Fig.\,14 for experimental confirmation).

\textbf{Row 2 (loading powers)}: shows the instantaneous $\PHi{i}$ during
the transient. Conventional: $\PHi{i}$ changes slowly as the PI integrates.
Proposed: $\PHi{i}$ is set to the maximum achievable value immediately at
$t = 0$, then pulled back to zero as the voltages converge.

\textbf{Row 3 (modulation indices)}: shows the per-cell $\Mi$ changing during
the transient. Conventional: small incremental changes. Proposed: large step
changes at $t = 0$ (maximum correction), then gradual return to steady-state
values.
\end{physbox}

\clearpage

\subsection{Fig.\,11 --- Simulation: Maximum Performance Comparison}

\begin{figure}[H]
    \centering
    \begin{subfigure}[b]{0.48\textwidth}
        \centering
        \includegraphics[width=\textwidth]{P3_Fig11a_maxperf_conventional.png}
        \caption{Maximum performance: conventional method.}
        \label{fig:P3Fig11a}
    \end{subfigure}
    \hfill
    \begin{subfigure}[b]{0.48\textwidth}
        \centering
        \includegraphics[width=\textwidth]{P3_Fig11b_maxperf_proposed.png}
        \caption{Maximum performance: proposed method.}
        \label{fig:P3Fig11b}
    \end{subfigure}
    \caption{Fig.\,11 --- Maximum achievable balancing performance comparison
             between conventional and proposed methods.}
    \label{fig:P3Fig11}
\end{figure}

\clearpage
% ============================================================
\section{Experimental Validation}
\label{sec:experimental}
% ============================================================

\subsection{The Prototype (Fig.\,12)}

\figentry{fig:P3Fig12}{P3_Fig12_prototype_photo.png}%
{Fig.\,12 --- Three-cell CHB prototype used for experimental validation.
 Specifications: $n = 3$ cells, grid voltage 230\,V$_{\rm rms}$ (single-phase),
 DC-link voltage per cell: target $\Vdc = 180$\,V,
 DC-link capacitance: \SI{1100}{\micro\farad} per cell.}{0.7}

The prototype implements all three loops (outer, inner, balancing) on a
DSP. The three cells have independently controllable DC-link sources (separate
power supplies) to simulate unequal source powers.

Steady-state and dynamic tests are performed by suddenly changing the source
powers of one or more cells and measuring how fast the algorithm restores balance.

\subsection{Steady-State Results (Fig.\,13)}

\begin{figure}[H]
    \centering
    \begin{subfigure}[b]{0.48\textwidth}
        \centering
        \includegraphics[width=\textwidth]{P3_Fig13a_exp_cond1_strip1.png}
        \caption{Condition 1, waveform strip 1.}
    \end{subfigure}
    \hfill
    \begin{subfigure}[b]{0.48\textwidth}
        \centering
        \includegraphics[width=\textwidth]{P3_Fig13b_exp_cond1_strip2.png}
        \caption{Condition 1, waveform strip 2.}
    \end{subfigure}
    \\[4pt]
    \begin{subfigure}[b]{0.48\textwidth}
        \centering
        \includegraphics[width=\textwidth]{P3_Fig13c_exp_cond2_strip1.png}
        \caption{Condition 2, waveform strip 1.}
    \end{subfigure}
    \hfill
    \begin{subfigure}[b]{0.48\textwidth}
        \centering
        \includegraphics[width=\textwidth]{P3_Fig13d_exp_cond2_strip2.png}
        \caption{Condition 2, waveform strip 2.}
    \end{subfigure}
    \\[4pt]
    \begin{subfigure}[b]{0.48\textwidth}
        \centering
        \includegraphics[width=\textwidth]{P3_Fig13e_exp_cond3_strip1.png}
        \caption{Condition 3, waveform strip 1.}
    \end{subfigure}
    \hfill
    \begin{subfigure}[b]{0.48\textwidth}
        \centering
        \includegraphics[width=\textwidth]{P3_Fig13f_exp_cond3_strip2.png}
        \caption{Condition 3, waveform strip 2.}
    \end{subfigure}
    \caption{Fig.\,13 --- Steady-state experimental results under three
             different power imbalance conditions. Each condition shows two
             oscilloscope strips: DC-link voltages (strip 1) and grid current
             with cell output voltages (strip 2).}
    \label{fig:P3Fig13}
\end{figure}

\begin{physbox}[Reading Fig.\,13]
Each condition applies a different power imbalance across the three cells.
The oscilloscope strips confirm:
\begin{enumerate}
    \item All three DC-link voltages remain at 180\,V despite unequal source
          powers --- the algorithm maintains balance in steady state.
    \item The grid current waveform remains sinusoidal --- the modulation
          changes (clamped vs.\ non-clamped assignment) do not introduce
          visible current distortion at the grid terminal.
    \item The cell output voltages show the expected staircase pattern (3-level
          approximation to a sinusoid).
\end{enumerate}
\end{physbox}

\subsection{Dynamic Response (Fig.\,14)}

\begin{figure}[H]
    \centering
    \begin{subfigure}[b]{0.48\textwidth}
        \centering
        \includegraphics[width=\textwidth]{P3_Fig14a_dynamic_conventional.png}
        \caption{Conventional PI: settling time $\sim$1850\,ms.}
        \label{fig:P3Fig14a}
    \end{subfigure}
    \hfill
    \begin{subfigure}[b]{0.48\textwidth}
        \centering
        \includegraphics[width=\textwidth]{P3_Fig14b_dynamic_proposed.png}
        \caption{Proposed algorithm: settling time $\sim$140\,ms.}
        \label{fig:P3Fig14b}
    \end{subfigure}
    \caption{Fig.\,14 --- Dynamic response comparison. A step change in power
             imbalance is applied at $t = 0$. The proposed algorithm restores
             balance 13$\times$ faster than the conventional PI method.}
    \label{fig:P3Fig14}
\end{figure}

\begin{physbox}[The 13$\times$ Speedup --- Where Does It Come From?]
\textbf{Conventional PI}: settling time 1850\,ms = 92.5 fundamental cycles
at 50\,Hz. The PI integrates over many cycles with a limited correction
per cycle (small $\Delta M$).

\textbf{Proposed algorithm}: settling time 140\,ms = 7 fundamental cycles.
The algorithm applies maximum-possible loading power correction from the
first cycle. The correction is computed from the exact loading power
expressions, not accumulated from an integral.

\medskip
The speedup factor $\approx 13\times$ reflects the ratio of maximum achievable
loading power between the two methods:
\[
    \text{speedup} \approx \frac{|\PH|_{\max}^{\mathrm{proposed}}}{|\PH|_{\max}^{\mathrm{conv}}}
\]
as shown analytically in Fig.\,9.

\medskip
The practical implication: in a grid-tied CHB inverter subject to rapid
irradiance transients (clouds, partial shading), the proposed method can
rebalance the DC-link voltages within 3--4 fundamental cycles --- fast enough
to prevent over-voltage even during rapid source power changes.
\end{physbox}

\clearpage
% ============================================================
\section{Key Takeaways}
\label{sec:takeaways}
% ============================================================

\begin{enumerate}
    \item \textbf{The root cause of DC-link imbalance} is that each cell has
          a different source power, but the same AC current flows through all
          cells. Without active balancing, the voltages diverge.

    \item \textbf{Loading power $\PH$} is the net power flowing into (or out of)
          each cell's DC-link capacitor. Controlling $\PH$ directly controls the
          rate of voltage change of $\Vdci{i}$.

    \item \textbf{Clamped modulation} provides an extra degree of freedom (the
          clamping angle $\phic$) for setting $\PH$. Non-clamped modulation has
          only one degree of freedom ($\Mi$).

    \item \textbf{The sorting algorithm} assigns clamped mode to cells needing
          discharge (high loading power drain) and non-clamped mode to cells
          needing charge (high loading power input), then solves for the exact
          $\Mi$ and $\phic$ values.

    \item \textbf{Speed advantage}: the proposed method applies maximum loading
          power correction immediately (one fundamental cycle), vs.\ the PI
          which integrates slowly. Result: 13$\times$ faster settling time
          (140\,ms vs.\ 1850\,ms).

    \item \textbf{Steady-state}: in balanced steady state, the algorithm reduces
          to conventional modulation with no performance penalty.

    \item \textbf{Grid current quality}: the modulation changes between cells
          (clamped vs.\ non-clamped) do not significantly distort the total
          grid current, since the sum of cell voltages still approximates a
          sinusoid.
\end{enumerate}

\end{document}
