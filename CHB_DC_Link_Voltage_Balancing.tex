% ============================================================
%  DC-Link Voltage Balancing Modulation for Cascaded H-Bridge Converters
%  Paper 3 Reference Notes
%  Source: [Vasishta et al.], IEEE Access
%  Figures stored in: figures/
%  Naming convention: P3_Fig[N]_page[XX].png  (caption-extracted at 200 DPI)
% ============================================================
\documentclass[12pt, a4paper]{article}

% ---- core packages ----
\usepackage[utf8]{inputenc}
\usepackage[T1]{fontenc}
\usepackage{amsmath, amssymb, amsthm}
\usepackage{bm}
\usepackage{siunitx}

% ---- figures ----
\usepackage{graphicx}
\graphicspath{{figures/}}
\usepackage{float}
\usepackage{caption}
\usepackage{subcaption}

% ---- layout ----
\usepackage[a4paper, top=2.5cm, bottom=2.5cm, left=3cm, right=2.8cm]{geometry}
\usepackage{parskip}
\setlength{\parskip}{6pt}
\setlength{\parindent}{0pt}

% ---- hyperlinks ----
\usepackage[colorlinks=true, linkcolor=blue!60!black, citecolor=blue!50!black]{hyperref}
\usepackage{bookmark}
\makeatletter
\pdfstringdefDisableCommands{%
    \def\vspace#1{}%
    \def\rule#1#2{}%
    \def\color#1{}%
    \let\leavevmode@ifvmode\relax%
    \def\,{ }%
}
\makeatother

% ---- styled boxes ----
\usepackage{tcolorbox}
\tcbuselibrary{skins, breakable}

% ---- tables ----
\usepackage{booktabs}
\usepackage{array}
\usepackage{longtable}

% ---- section formatting ----
\usepackage{titlesec}
\usepackage{xcolor}
\definecolor{chbblue}{RGB}{10, 55, 120}
\definecolor{chborange}{RGB}{190, 70, 5}
\definecolor{chbgreen}{RGB}{8, 95, 38}
\definecolor{chbred}{RGB}{170, 25, 25}
\definecolor{lightblue}{RGB}{216, 232, 252}
\definecolor{lightorange}{RGB}{255, 244, 220}
\definecolor{lightgreen}{RGB}{216, 247, 228}
\definecolor{lightred}{RGB}{255, 228, 228}
\definecolor{lightgray}{RGB}{245, 245, 245}

\titleformat{\section}{\large\bfseries\color{chbblue}}{{\thesection.}}{0.5em}{}
            [\vspace{-4pt}\rule{\textwidth}{0.5pt}]
\titleformat{\subsection}{\normalsize\bfseries\color{chbblue!80!black}}{{\thesubsection}}{0.5em}{}

% ---- tcolorbox styles ----
\newtcolorbox{circuitbox}[1][Circuit Description]{
    colback=lightblue, colframe=chbblue, coltitle=white,
    fonttitle=\bfseries\small, title={#1},
    breakable, enhanced, left=4pt, right=4pt, top=4pt, bottom=4pt}

\newtcolorbox{statebox}[1][State Variables \& Parameters]{
    colback=lightorange, colframe=chborange, coltitle=white,
    fonttitle=\bfseries\small, title={#1},
    breakable, enhanced, left=4pt, right=4pt, top=4pt, bottom=4pt}

\newtcolorbox{eqnbox}[1][Governing Equations]{
    colback=lightgreen, colframe=chbgreen, coltitle=white,
    fonttitle=\bfseries\small, title={#1},
    breakable, enhanced, left=4pt, right=4pt, top=4pt, bottom=4pt}

\newtcolorbox{physbox}[1][Physical Interpretation]{
    colback=lightgray, colframe=gray!60, coltitle=white,
    fonttitle=\bfseries\small, title={#1},
    breakable, enhanced, left=4pt, right=4pt, top=4pt, bottom=4pt}

\newtcolorbox{warningbox}[1][Watch Out]{
    colback=lightred, colframe=chbred, coltitle=white,
    fonttitle=\bfseries\small, title={#1},
    breakable, enhanced, left=4pt, right=4pt, top=4pt, bottom=4pt}

% ---- custom commands ----
\newcommand{\vg}{v_{g}}
\newcommand{\Vg}{V_{g}}
\newcommand{\ig}{i_{g}}
\newcommand{\Ig}{I_{g}}
\newcommand{\omg}{\omega}
\newcommand{\phig}{\phi_{g}}
\newcommand{\Vdc}{V_{dc}}
\newcommand{\Vdci}[1]{V_{dc,#1}}
\newcommand{\Vtot}{V_{dc,\mathrm{tot}}}
\newcommand{\Mi}{M_{i}}
\newcommand{\phic}{\varphi}
\newcommand{\vref}{v_{\mathrm{ref}}}
\newcommand{\PH}{P_{H}}
\newcommand{\PHi}[1]{P_{H,#1}}
\newcommand{\PHnc}{P_{H}^{\mathrm{nc}}}
\newcommand{\PHcl}{P_{H}^{\mathrm{cl}}}
\newcommand{\figref}[1]{Figure~\ref{#1}}

% ---- figure entry macro ----
\newcommand{\figentry}[4]{%
    \begin{figure}[H]
        \centering
        \includegraphics[width=#4\textwidth]{#2}
        \caption{#3}
        \label{#1}
    \end{figure}%
}

% ============================================================
\title{\textbf{DC-Link Voltage Balancing Modulation}\\[4pt]
       \textbf{for Cascaded H-Bridge Converters}\\[6pt]
       \large Paper 3 --- Circuit Topology, Control Architecture,\\
       \large Loading Power Theory, and Sorting Algorithm\\[4pt]
       \normalsize IEEE Access}
\date{\today}
% ============================================================

\begin{document}
\maketitle
\tableofcontents
\clearpage

% ============================================================
\section{How to Use This Document}
% ============================================================

Each section follows this standard structure:

\begin{enumerate}
    \item \textbf{Figure} --- extracted directly from the paper (cropped by caption detection)
    \item \textbf{Circuit Description} (blue box) --- what components are present
          and how they connect
    \item \textbf{State Variables \& Parameters} (orange box) --- every symbol,
          with units and physical meaning
    \item \textbf{Governing Equations} (green box) --- equations derived from first principles
    \item \textbf{Physical Interpretation} (gray box) --- what the result means:
          energy flow, design trade-offs, why it matters
\end{enumerate}

\clearpage

% ============================================================
\section{The Problem: Why Does the DC-Link Become Unbalanced?}
\label{sec:problem}
% ============================================================

\subsection{Start Here: What Is a Cascaded H-Bridge?}

Before the equations, the physical picture.

A \textbf{Cascaded H-Bridge (CHB)} inverter is built by stacking multiple
H-bridge converter cells in series. Each cell has:
\begin{itemize}
    \item its own isolated DC-link capacitor ($C_{dc,i}$, voltage $\Vdci{i}$),
    \item four switches forming a full H-bridge,
    \item a separate DC source (solar panel, battery, or simply a capacitor).
\end{itemize}
The output voltages of all cells add up in series. If there are $n$ cells,
the total AC output is:
\[
    \vg = \sum_{i=1}^{n} v_{i}(t)
\]

\textbf{Why use CHB?} High AC voltage from low-voltage DC sources without a
transformer. Each cell contributes a small voltage step; their staircase sum
approximates a sinusoid with low harmonic distortion.

\textbf{The problem:} Real DC sources are not identical. Solar panels on
different roof faces see different irradiance. Without balancing, the DC-link
voltages $\Vdci{i}$ diverge over time --- some cells over-charge, others under-charge.

\figentry{fig:P3Fig1}{P3_Fig1_page01.png}%
{Fig.\,1 --- CHB topology: $n$ H-bridge cells in series, each with its own
 isolated DC-link capacitor. The source powers $P_{1H},\,P_{2H},\ldots,P_{nH}$
 are generally unequal. The AC outputs are summed in series and connected
 to the grid through filter inductance $L_f$.}{0.65}

\begin{circuitbox}[Circuit Description --- CHB Topology (Fig.\,1)]
Each cell $i$ in the CHB string contains:
\begin{itemize}
    \item \textbf{DC-link capacitor} $C_{dc,i}$: stores DC energy at voltage $\Vdci{i}$.
    \item \textbf{H-bridge switches}: four IGBTs chopping the DC voltage into AC
          via pulse-width modulation.
    \item \textbf{DC source}: delivers power $P_{iH}$ into the DC link.
\end{itemize}
The AC outputs of all cells connect in series. The common grid current $\ig$
flows through every cell --- but each cell's DC voltage $\Vdci{i}$ is
independent. This is the structural source of the balancing challenge.
\end{circuitbox}

\begin{statebox}
\renewcommand{\arraystretch}{1.3}
\begin{tabular}{p{0.12\linewidth} p{0.32\linewidth} p{0.46\linewidth}}
\toprule
\textbf{Symbol} & \textbf{Quantity} & \textbf{Role / Typical Value} \\
\midrule
$n$        & Number of H-bridge cells  & 3 in prototype (230\,V$_{\rm rms}$ grid) \\
$\Vdci{i}$ & DC-link voltage, cell $i$ & Target: 180\,V each \\
$\Vtot$    & $\sum_{i=1}^{n}\Vdci{i}$  & Controlled by outer loop \\
$\PHi{i}$  & Loading power, cell $i$   & Net power from DC source into $C_{dc,i}$ \\
$\ig$      & Grid current              & Same through all cells \\
$L_{f}$    & Filter inductance         & Grid-side series inductor \\
\bottomrule
\end{tabular}
\end{statebox}

\begin{physbox}[Physical Interpretation --- Why Voltages Diverge]
At steady state, each cell's capacitor voltage is constant: power in from the
DC source equals power out to the AC grid. If cell\,1 sources 500\,W and
cell\,2 sources 200\,W, but both must carry the same AC current $\ig$, there
is a mismatch. Without correction, cell\,1 over-charges and cell\,2 under-charges
indefinitely. The balancing controller must redistribute how much AC power each
cell contributes so that every capacitor reaches equilibrium at $\Vdc$.
\end{physbox}

\clearpage
% ============================================================
\section{The Control Architecture}
\label{sec:control}
% ============================================================

\figentry{fig:P3Fig2}{P3_Fig2_page02.png}%
{Fig.\,2 --- Three-loop control architecture: (1) outer $\Vtot$ loop sets
 grid current reference amplitude; (2) inner $\ig$ loop tracks the reference;
 (3) balancing loop assigns per-cell modulation strategy (clamped vs.
 non-clamped) and computes individual modulation indices.}{0.90}

The control has three layers, slow to fast:

\subsection{Outer Loop: Total DC-Link Voltage Control}
A PI controller compares $\Vtot = \sum_{i}\Vdci{i}$ to the reference $nV_{dc,\mathrm{ref}}$.
Output: grid current reference amplitude $\Ig^{*}$. Timescale: slow.

\subsection{Inner Loop: Grid Current Control}
A fast current controller (PI or PR) generates the total modulation reference
$\vref(t)$ that is then split among the $n$ cells. Timescale: one to a few
switching periods.

\subsection{Balancing Loop: Per-Cell Voltage Control}
This is the key contribution. The balancing loop assigns each cell a
modulation strategy (clamped or non-clamped) and computes $\Mi$ and clamping
angle $\phic$ so that the \textbf{loading power} $\PHi{i}$ of each cell is
set to exactly the value needed to charge or discharge $C_{dc,i}$ back to $V_{dc,\mathrm{ref}}$.

\begin{physbox}[Why Three Loops?]
The outer loop controls the total energy across all cells --- it does not
care which cell holds it. The balancing loop redistributes energy between
cells through the modulation alone, without adding hardware. The inner current
loop ensures redistribution does not distort the grid current.
\end{physbox}

\clearpage
% ============================================================
\section{Conventional Balancing: The PI-Based Approach}
\label{sec:conventional}
% ============================================================

\figentry{fig:P3Fig3}{P3_Fig3_page02.png}%
{Fig.\,3 --- Conventional PI balancing algorithm. Each cell $i$ has its own
 PI controller that compares $\Vdci{i}$ to the reference and outputs a
 small modulation correction $\Delta M_i$.}{0.85}

Each cell gets a PI controller watching the voltage error
$\varepsilon_i = \Vdci{i} - V_{dc,\mathrm{ref}}$:
\[
    \Delta M_i = \left(K_P + \frac{K_I}{s}\right) \varepsilon_i
\]
This correction $\Delta M_i$ is added to the shared base modulation reference.
A cell with over-voltage gets a larger modulation index (extracts more power,
pulls voltage down). A cell with under-voltage gets a smaller modulation index
(extracts less power, allows charging).

\begin{warningbox}[Why Is the Conventional Method Slow?]
The PI generates only a small $\Delta M_i$ per cycle --- otherwise the grid
current distorts. This small correction acts against the large stored energy
in $C_{dc,i}$, forcing the PI to integrate slowly over many cycles.
Measured settling time: $\sim$\,1850\,ms. The proposed method bypasses
this by \emph{directly setting} the loading power --- a fundamentally
faster degree of freedom.
\end{warningbox}

\clearpage
% ============================================================
\section{Key Concept: What Is Loading Power?}
\label{sec:loading_power}
% ============================================================

\subsection{Power at Each Cell}

Each cell $i$ produces AC output voltage $v_i(t)$. The instantaneous power
it delivers to the grid is $p_i(t) = v_i(t) \cdot \ig(t)$.

With sinusoidal grid current $\ig(t) = \Ig\sin(\omg t)$ and non-clamped
modulation $v_i(t) = \Vdci{i}\Mi\sin(\omg t)$:
\[
    p_i(t) = \frac{\Vdci{i}\Mi\Ig}{2}
             \bigl[\cos(\phig) - \cos(2\omg t - \phig)\bigr]
\]

The \emph{average} (real) power delivered to the grid:
\[
    \langle p_i \rangle = \frac{\Vdci{i}\Mi\Ig}{2}\cos(\phig)
\]

\subsection{Loading Power Definition}

The \textbf{loading power} $\PH$ is the net average power flowing into the
DC-link capacitor:
\[
    \PH = P_{i,\mathrm{src}} - \langle p_i \rangle
\]

\begin{itemize}
    \item $\PH > 0$: capacitor is charging --- $\Vdci{i}$ \emph{rises}.
    \item $\PH < 0$: capacitor is discharging --- $\Vdci{i}$ \emph{falls}.
    \item $\PH = 0$: capacitor voltage is constant (balanced).
\end{itemize}

\begin{physbox}[Analogy: Loading Power as a Water Tap]
The DC-link capacitor is a water tank. The DC source is an inlet pipe (fills
at rate $P_{i,\mathrm{src}}$). The H-bridge is a drain (empties at rate
$\langle p_i\rangle$). Loading power = inlet minus drain = net level change.

To raise voltage: reduce the drain (smaller $\Mi$ $\Rightarrow$ less AC power extracted).
To lower voltage: increase the drain (larger $\Mi$ $\Rightarrow$ more AC power extracted).

The key insight: the modulation strategy (clamped vs.\ non-clamped) directly
sets the drain rate --- so by choosing the modulation, $\PH$ is set immediately,
without waiting for a PI to integrate.
\end{physbox}

\clearpage
% ============================================================
\section{Clamped vs.\ Non-Clamped Modulation}
\label{sec:clamped}
% ============================================================

\figentry{fig:P3Fig7}{P3_Fig7_page04.png}%
{Fig.\,7 --- Clamped modulation reference signal. The sinusoidal reference
 is held at $\pm 1$ during the clamped intervals (flat-top sections near
 the peaks). The clamping angle $\phic$ controls the fraction of the cycle
 that is clamped. Larger $\phic$ $\Rightarrow$ wider flat-top regions.}{0.80}

\subsection{Non-Clamped (Standard Sinusoidal) Modulation}

Reference: $\vref^{\mathrm{nc}}(t) = \Mi\sin(\omg t)$.
The H-bridge switches continuously; the duty cycle varies smoothly. $\Mi \leq 1$
at all times. Average power to grid: $\frac{\Vdci{i}\Mi\Ig}{2}\cos(\phig)$.

\subsection{Clamped Modulation}

During portions of the cycle where the sinusoidal reference is near $\pm 1$,
the reference is clamped to exactly $\pm 1$:
\[
    \vref^{\mathrm{cl}}(t) =
    \begin{cases}
        +1 & \text{if } \Mi\sin(\omg t) \geq \cos(\phic) \\
        -1 & \text{if } \Mi\sin(\omg t) \leq -\cos(\phic) \\
        \Mi\sin(\omg t) & \text{otherwise}
    \end{cases}
\]
During clamped intervals: one switch is fully on, no switching losses, and
the cell output is exactly $\pm\Vdci{i}$.

\begin{statebox}
\renewcommand{\arraystretch}{1.3}
\begin{tabular}{p{0.13\linewidth} p{0.35\linewidth} p{0.44\linewidth}}
\toprule
\textbf{Symbol} & \textbf{Quantity} & \textbf{Notes} \\
\midrule
$\Mi$   & Modulation index, cell $i$ & $0 \leq \Mi \leq 1$ \\
$\phic$ & Clamping angle & $0$ (no clamping) to $\pi/2$ (fully clamped) \\
$\phig$ & Power factor angle & Fixed by load \\
$\PHnc$ & Non-clamped loading power & Function of $\Mi$ only \\
$\PHcl$ & Clamped loading power & Function of $\Mi$ and $\phic$ \\
\bottomrule
\end{tabular}
\end{statebox}

\begin{physbox}[Why Does Clamping Change Loading Power?]
When clamped, the cell output is a DC level $(\pm\Vdci{i})$ during the
clamped intervals rather than a PWM fraction. The energy extracted per cycle
changes. By varying $\phic$, the fraction of the cycle that is clamped
changes --- giving an extra degree of freedom ($\phic$) for setting $\PH$.
Non-clamped cells have only $\Mi$ as a free variable. This extra freedom
is what the sorting algorithm exploits.
\end{physbox}

\clearpage
% ============================================================
\section{The Proposed Sorting Algorithm}
\label{sec:algorithm}
% ============================================================

\figentry{fig:P3Fig4}{P3_Fig4_page03.png}%
{Fig.\,4 --- Proposed sorting algorithm flowchart. Cells are sorted by
 their required loading power correction direction, then assigned clamped
 (discharge) or non-clamped (charge) modulation mode. The modulation
 parameters $\Mi$ and $\phic$ are computed from closed-form loading
 power expressions.}{0.75}

\subsection{Algorithm Steps}

\textbf{Step 1 --- Compute voltage errors.}
$\varepsilon_i = V_{dc,\mathrm{ref}} - \Vdci{i}$.
Positive: cell is under-voltage (needs charging). Negative: over-voltage (needs discharging).

\medskip
\textbf{Step 2 --- Sort by error magnitude.}
Rank cells by $|\varepsilon_i|$ descending. The furthest-from-target cell
gets the most aggressive correction.

\medskip
\textbf{Step 3 --- Assign modulation mode.}
\begin{itemize}
    \item Over-voltage cells ($\varepsilon_i < 0$) $\rightarrow$ \textbf{clamped mode}
          (higher loading power drain, faster discharge).
    \item Under-voltage cells ($\varepsilon_i > 0$) $\rightarrow$ \textbf{non-clamped mode}
          (maximize loading power input, faster charge).
\end{itemize}

\medskip
\textbf{Step 4 --- Compute $\Mi$ and $\phic$.}
Using the loading power expressions (Section~\ref{sec:loading_power_derivation}),
find the exact values that drive $\varepsilon_i \to 0$ in minimum time.

\medskip
\textbf{Step 5 --- Verify total voltage constraint.}
\[
    \sum_{i=1}^{n} \Vdci{i}\Mi\sin(\omg t + \theta_i)
    = \Vtot \cdot M_{\mathrm{ref}} \cdot \sin(\omg t)
\]
Adjust $\Mi$ values if needed to satisfy this constraint.

\begin{physbox}[Why Is This Faster Than PI?]
PI changes $\Delta M_i$ by a small increment each cycle, limited by stability.
The sorting algorithm directly \emph{sets} the loading power target and
computes the exact $\Mi$, $\phic$ that deliver it. This is the difference
between slowly adjusting a tap (PI) vs.\ opening a valve all the way and
computing exactly when to close it. Result: 140\,ms vs.\ 1850\,ms --- a
factor of 13$\times$ faster.
\end{physbox}

\clearpage
% ============================================================
\section{Two-Cell Case Study}
\label{sec:two_cell}
% ============================================================

\figentry{fig:P3Fig5}{P3_Fig5_page03.png}%
{Fig.\,5 --- Two-cell case study showing the feasible clamped and
 non-clamped operating regions in the $(M,\,\phic)$ plane. The algorithm
 selects the operating point that delivers the required loading power
 while satisfying the total modulation constraint.}{0.85}

With $n = 2$ cells, suppose cell\,1 is over-voltage and cell\,2 is under-voltage.

The algorithm assigns cell\,1 to \textbf{clamped mode} and cell\,2 to
\textbf{non-clamped mode}. The clamped mode gives cell\,1 a higher loading
power drain (faster discharge); the non-clamped mode maximizes loading power
input into cell\,2 (faster charge). The values of $M_1$, $M_2$, $\phic$ are
solved so that both cells reach $V_{dc,\mathrm{ref}}$ simultaneously while
the total modulation sum remains correct.

\clearpage
% ============================================================
\section{Four-Cell Extension: Five Operating Cases}
\label{sec:four_cell}
% ============================================================

\figentry{fig:P3Fig6}{P3_Fig6_page04.png}%
{Fig.\,6 --- Five distinct cases for four cells, depending on how many cells
 need discharge vs.\ charge. Case I: all non-clamped. Cases II--IV: mixed
 assignments. Case V: all clamped.}{0.75}

\begin{center}
\renewcommand{\arraystretch}{1.3}
\begin{tabular}{clll}
\toprule
\textbf{Case} & \textbf{Discharge} & \textbf{Charge} & \textbf{Mode assignment} \\
\midrule
I   & 0 & 4 & All non-clamped \\
II  & 1 & 3 & 1 clamped + 3 non-clamped \\
III & 2 & 2 & 2 clamped + 2 non-clamped \\
IV  & 3 & 1 & 3 clamped + 1 non-clamped \\
V   & 4 & 0 & All clamped \\
\bottomrule
\end{tabular}
\end{center}

\begin{physbox}[Symmetry of the Five Cases]
Cases I and V are symmetric: all cells on the same side of the target.
Cases II--IV are the most common during transients. In each case, the sorting
algorithm solves the loading power constraints to find the optimal
$\{\Mi,\phic\}$ set for that case.
\end{physbox}

\clearpage
% ============================================================
\section{Loading Power Derivation}
\label{sec:loading_power_derivation}
% ============================================================

\subsection{Non-Clamped Loading Power $\PHnc$}

\[
    \boxed{\PHnc(\Mi)
    = P_{i,\mathrm{src}} - \frac{\Vdci{i}\Mi\Ig}{2}\cos(\phig)}
\]

To increase $\PHnc$ (charge the capacitor): decrease $\Mi$.
One free variable: $\Mi$.

\subsection{Clamped Loading Power $\PHcl$}

The clamped output voltage has non-sinusoidal content; the average power
integral over one cycle gives:
\[
    \boxed{\PHcl(\Mi,\phic)
    = P_{i,\mathrm{src}} - \frac{\Vdci{i}\Ig}{2\pi}
      \bigl[\Mi\pi\cos(\phig) + 2\cos(\phic)\sin(\phig)\bigr]}
\]

Two free variables: $\Mi$ and $\phic$. This is the extra degree of freedom
that allows the clamped mode to deliver more loading power than the
non-clamped mode.

\begin{warningbox}[Verify Against Paper]
The coefficient form above captures the correct structure. Verify the exact
integral limits and trigonometric identities against the paper's derivation
before using these equations in calculations.
\end{warningbox}

\begin{eqnbox}[Summary: Loading Power Expressions]
\textbf{Non-clamped} (1 free variable: $\Mi$):
\[
    \PHnc = P_{i,\mathrm{src}} - \frac{\Vdci{i}\Mi\Ig}{2}\cos(\phig)
\]

\textbf{Clamped} (2 free variables: $\Mi$, $\phic$):
\[
    \PHcl = P_{i,\mathrm{src}} - \frac{\Vdci{i}\Ig}{2\pi}
            \bigl[\Mi\pi\cos(\phig) + 2\cos(\phic)\sin(\phig)\bigr]
\]

\textbf{Constraint} (total grid voltage):
\[
    \sum_{i=1}^{n}\Vdci{i}\Mi = \Vtot\cdot M_{\mathrm{ref}}
\]
\end{eqnbox}

\clearpage
% ============================================================
\section{Loading Power Surfaces (Fig.\,8)}
\label{sec:3d_surface}
% ============================================================

\figentry{fig:P3Fig8}{P3_Fig8_page04.png}%
{Fig.\,8 --- 3D loading power surfaces as functions of modulation index $M$
 and clamping angle $\phic$. Clamped mode surface (left): $\PHcl(M,\phic)$
 shows the two-variable freedom. Non-clamped mode (right): $\PHnc(M)$ is a
 1D curve showing the single-variable constraint. The algorithm uses these
 surfaces to look up $(M,\phic)$ pairs for any required $\PH$.}{0.90}

\begin{physbox}[Reading the Loading Power Surface]
\textbf{Clamped surface axes:}
x = modulation index $M$; y = clamping angle $\phic$; z = loading power $\PH$.

At $\phic = 0$: clamped surface matches the non-clamped curve (no clamping
= pure sinusoidal). As $\phic$ increases, the achievable loading power range
expands. The maximum loading power (most aggressive correction) lies at the
outer edge of the clamped surface. This expanded range is why the proposed
method is faster than the PI method.
\end{physbox}

\clearpage
% ============================================================
\section{Performance Comparison: Maximum Loading Power (Fig.\,9)}
\label{sec:performance_comparison}
% ============================================================

\figentry{fig:P3Fig9}{P3_Fig9_page04.png}%
{Fig.\,9 --- Maximum achievable loading power $|\PH|_{\max}$ vs.\ modulation
 index $M$, comparing proposed sorting algorithm (solid) to conventional
 PI method (dashed). The proposed method achieves higher maximum loading
 power across the full range of $M$, enabling faster balancing.}{0.75}

\begin{physbox}[What Fig.\,9 Shows and Why It Matters]
A higher $|\PH|_{\max}$ means more energy transferred per cycle, so any
voltage error is corrected faster. The proposed algorithm is better because:
\begin{enumerate}
    \item Clamped cells use both $M$ and $\phic$, giving a wider range of
          achievable $|\PH|$.
    \item The sorting step always assigns the most favorable mode to each cell.
    \item The PI is limited to small $\Delta M$ corrections, capping its
          achievable loading power.
\end{enumerate}
The speedup factor $\approx 13\times$ reflects the ratio
$|\PH|_{\max}^{\mathrm{proposed}} / |\PH|_{\max}^{\mathrm{conv}}$.
\end{physbox}

\clearpage
% ============================================================
\section{Simulation Results (Figs.\,10 and 11)}
\label{sec:simulation}
% ============================================================

\figentry{fig:P3Fig10}{P3_Fig10_page05.png}%
{Fig.\,10 --- Simulation comparison (6-panel, $2\times3$ grid). Left column:
 conventional PI balancing. Right column: proposed sorting algorithm.
 Row 1: DC-link voltages $\Vdci{i}$. Row 2: loading powers $\PHi{i}$.
 Row 3: modulation indices $\Mi$.
 The proposed method converges in $\sim$140\,ms vs.\ $\sim$1850\,ms for
 the conventional method.}{0.90}

\begin{physbox}[How to Read Fig.\,10]
\textbf{Row 1 (DC-link voltages)}: convergence speed is the key metric.
Proposed: fast step-like convergence within 7 fundamental cycles.
Conventional: slow exponential approach over $\sim$92 cycles.

\textbf{Row 2 (loading powers)}: Proposed sets $\PHi{i}$ to maximum
immediately at $t=0$, then pulls back as voltages converge. Conventional:
$\PHi{i}$ changes slowly as the PI integrates.

\textbf{Row 3 (modulation indices)}: Proposed makes a large step change
at $t=0$ (full correction), then returns to steady-state values.
Conventional: only small incremental changes per cycle.
\end{physbox}

\figentry{fig:P3Fig11}{P3_Fig11_page06.png}%
{Fig.\,11 --- Maximum balancing performance comparison between conventional
 and proposed methods under the most demanding imbalance conditions.}{0.85}

\clearpage
% ============================================================
\section{Experimental Validation}
\label{sec:experimental}
% ============================================================

\figentry{fig:P3Fig12}{P3_Fig12_page06.png}%
{Fig.\,12 --- Three-cell CHB prototype. Specifications: $n=3$ cells,
 grid 230\,V$_{\rm rms}$ single-phase, DC-link target $\Vdc=180$\,V per cell,
 DC-link capacitance \SI{1100}{\micro\farad} per cell. Control implemented
 on a DSP.}{0.85}

\figentry{fig:P3Fig13}{P3_Fig13_page07.png}%
{Fig.\,13 --- Steady-state experimental results under three different power
 imbalance conditions. Each condition shows oscilloscope captures of
 DC-link voltages and grid current with cell output voltages. All three
 DC-link voltages are maintained at 180\,V despite unequal source powers;
 the grid current remains sinusoidal.}{0.90}

\figentry{fig:P3Fig14}{P3_Fig14_page08.png}%
{Fig.\,14 --- Dynamic response comparison. A step power imbalance is applied
 at $t=0$. Left: conventional PI, settling time $\sim$1850\,ms. Right:
 proposed sorting algorithm, settling time $\sim$140\,ms. Factor of
 $13\times$ improvement.}{0.85}

\begin{physbox}[The 13$\times$ Speedup --- Where Does It Come From?]
\textbf{Conventional PI}: 1850\,ms = 92 fundamental cycles at 50\,Hz.
The PI integrates with a small $\Delta M$ per cycle.

\textbf{Proposed algorithm}: 140\,ms = 7 fundamental cycles. Maximum-possible
loading power correction is applied from cycle one, computed directly from
the closed-form expressions --- no integration needed.

Practical implication: for a grid-tied CHB inverter subject to rapid
irradiance transients (clouds, partial shading), the proposed method can
rebalance DC-link voltages within 3--4 fundamental cycles --- fast enough
to prevent over-voltage even during rapid source changes.
\end{physbox}

\clearpage
% ============================================================
\section{Key Takeaways}
\label{sec:takeaways}
% ============================================================

\begin{enumerate}
    \item \textbf{Root cause}: different source powers + same AC current through
          all cells = diverging DC-link voltages.
    \item \textbf{Loading power $\PH$}: the net power into each cell's DC-link
          capacitor. Controlling $\PH$ directly controls $\dot{V}_{dc,i}$.
    \item \textbf{Clamped modulation}: adds a second free variable ($\phic$)
          for setting $\PH$. Non-clamped has only $\Mi$.
    \item \textbf{Sorting algorithm}: assigns clamped mode to over-voltage cells
          (high drain), non-clamped to under-voltage cells (high charge), then
          solves for exact $\Mi$, $\phic$ from closed-form expressions.
    \item \textbf{Speed}: 13$\times$ faster than conventional PI (140\,ms vs.\
          1850\,ms) because maximum loading power correction is applied immediately.
    \item \textbf{Steady-state}: no performance penalty when balanced --- algorithm
          reduces to conventional modulation.
    \item \textbf{Grid current quality}: clamped vs.\ non-clamped assignments
          do not significantly distort the total grid current.
\end{enumerate}

\end{document}
