% ============================================================
%  DC-Link Voltage Balancing Modulation for Cascaded H-Bridge Converters
%  Paper 3 Reference Notes
%  Source: [Vasishta et al.], IEEE Access
%  Figures stored in: figures/
%  Naming convention: DLVBMCHBC_Fig[N]_page[XX].png  (caption-extracted at 200 DPI)
% ============================================================
\documentclass[12pt, a4paper]{article}

% ---- core packages ----
\usepackage[utf8]{inputenc}
\usepackage[T1]{fontenc}
\usepackage{amsmath, amssymb, amsthm}
\usepackage{bm}
\usepackage{siunitx}

% ---- figures ----
\usepackage{graphicx}
\graphicspath{{figures/}}
\usepackage{float}
\usepackage{caption}
\usepackage{subcaption}

% ---- layout ----
\usepackage[a4paper, top=2.5cm, bottom=2.5cm, left=3cm, right=2.8cm]{geometry}
\usepackage{parskip}
\setlength{\parskip}{6pt}
\setlength{\parindent}{0pt}

% ---- hyperlinks ----
\usepackage[colorlinks=true, linkcolor=blue!60!black, citecolor=blue!50!black]{hyperref}
\usepackage{bookmark}
\makeatletter
\pdfstringdefDisableCommands{%
    \def\vspace#1{}%
    \def\rule#1#2{}%
    \def\color#1{}%
    \let\leavevmode@ifvmode\relax%
    \def\,{ }%
}
\makeatother

% ---- styled boxes ----
\usepackage{tcolorbox}
\tcbuselibrary{skins, breakable}

% ---- tables ----
\usepackage{booktabs}
\usepackage{array}
\usepackage{longtable}

% ---- section formatting ----
\usepackage{titlesec}
\usepackage{xcolor}
\definecolor{chbblue}{RGB}{10, 55, 120}
\definecolor{chborange}{RGB}{190, 70, 5}
\definecolor{chbgreen}{RGB}{8, 95, 38}
\definecolor{chbred}{RGB}{170, 25, 25}
\definecolor{lightblue}{RGB}{216, 232, 252}
\definecolor{lightorange}{RGB}{255, 244, 220}
\definecolor{lightgreen}{RGB}{216, 247, 228}
\definecolor{lightred}{RGB}{255, 228, 228}
\definecolor{lightgray}{RGB}{245, 245, 245}

\titleformat{\section}{\large\bfseries\color{chbblue}}{{\thesection.}}{0.5em}{}
            [\vspace{-4pt}\rule{\textwidth}{0.5pt}]
\titleformat{\subsection}{\normalsize\bfseries\color{chbblue!80!black}}{{\thesubsection}}{0.5em}{}

% ---- tcolorbox styles ----
\newtcolorbox{circuitbox}[1][Circuit Description]{
    colback=lightblue, colframe=chbblue, coltitle=white,
    fonttitle=\bfseries\small, title={#1},
    breakable, enhanced, left=4pt, right=4pt, top=4pt, bottom=4pt}

\newtcolorbox{statebox}[1][State Variables \& Parameters]{
    colback=lightorange, colframe=chborange, coltitle=white,
    fonttitle=\bfseries\small, title={#1},
    breakable, enhanced, left=4pt, right=4pt, top=4pt, bottom=4pt}

\newtcolorbox{eqnbox}[1][Governing Equations]{
    colback=lightgreen, colframe=chbgreen, coltitle=white,
    fonttitle=\bfseries\small, title={#1},
    breakable, enhanced, left=4pt, right=4pt, top=4pt, bottom=4pt}

\newtcolorbox{physbox}[1][Physical Interpretation]{
    colback=lightgray, colframe=gray!60, coltitle=white,
    fonttitle=\bfseries\small, title={#1},
    breakable, enhanced, left=4pt, right=4pt, top=4pt, bottom=4pt}

\newtcolorbox{warningbox}[1][Watch Out]{
    colback=lightred, colframe=chbred, coltitle=white,
    fonttitle=\bfseries\small, title={#1},
    breakable, enhanced, left=4pt, right=4pt, top=4pt, bottom=4pt}

% ---- custom commands ----
\newcommand{\vg}{v_{g}}
\newcommand{\Vg}{V_{g}}
\newcommand{\ig}{i_{g}}
\newcommand{\Ig}{I_{g}}
\newcommand{\omg}{\omega}
\newcommand{\phig}{\phi_{g}}
\newcommand{\Vdc}{V_{dc}}
\newcommand{\Vdci}[1]{V_{dc,#1}}
\newcommand{\Vtot}{V_{dc,\mathrm{tot}}}
\newcommand{\Mi}{M_{i}}
\newcommand{\phic}{\varphi}
\newcommand{\vref}{v_{\mathrm{ref}}}
\newcommand{\PH}{P_{H}}
\newcommand{\PHi}[1]{P_{H,#1}}
\newcommand{\PHnc}{P_{H}^{\mathrm{nc}}}
\newcommand{\PHcl}{P_{H}^{\mathrm{cl}}}
\newcommand{\figref}[1]{Figure~\ref{#1}}

% ---- figure entry macro ----
\newcommand{\figentry}[4]{%
    \begin{figure}[H]
        \centering
        \includegraphics[width=#4\textwidth]{#2}
        \caption{#3}
        \label{#1}
    \end{figure}%
}

% ============================================================
\title{\textbf{DC-Link Voltage Balancing Modulation}\\[4pt]
       \textbf{for Cascaded H-Bridge Converters}\\[6pt]
       \large Paper 3 --- Circuit Topology, Control Architecture,\\
       \large Loading Power Theory, and Sorting Algorithm\\[4pt]
       \normalsize IEEE Access}
\date{\today}
% ============================================================

\begin{document}
\maketitle
\tableofcontents
\clearpage

% ============================================================
\setcounter{section}{-1}
\section{Before You Begin --- The Big Picture}
\label{sec:bigpicture}
% ============================================================

\subsection{What Is This Document?}

This document is a self-contained reference guide to the paper:

\begin{center}
\textit{``DC-Link Voltage Balancing Modulation for Cascaded H-Bridge Converters''}\\
Y.\,Ko, A.\,Tcai, and M.\,Liserre, \textit{IEEE Access}, 2021.
\end{center}

It walks through every figure, every equation, and every design decision in
Feynman style --- starting from the simplest possible picture and building up
to the full algorithm. No derivation is skipped; no equation is left unexplained.

\textbf{What you will understand by the end:}
\begin{itemize}
    \item Why a Cascaded H-Bridge (CHB) inverter needs DC-link voltage balancing.
    \item What ``loading power'' is and why it is the right quantity to control.
    \item Why clamped modulation is faster than conventional PI control for balancing.
    \item How to compute the exact modulation parameters to balance any imbalance
          in minimum time.
\end{itemize}

\subsection{The Simplest Possible Description of a CHB}

\textbf{Start with a single H-bridge cell.} It has two ports:

\begin{itemize}
    \item \textbf{DC input port:} an isolated DC-link capacitor at voltage
          $\Vdci{i}$, continuously replenished by a DC source (a PV string,
          a battery pack, or a rectifier) delivering power $P_{iH}$.
          This port does not connect electrically to any other cell.
    \item \textbf{AC output port:} two terminals that the four transistors
          switch at high frequency. The average voltage across these terminals
          over one switching period is $v_i(t) = \Vdci{i} \cdot \Mi\sin(\omg t)$,
          where $\Mi$ is the modulation index the controller sets.
\end{itemize}

The H-bridge is therefore a controlled DC-to-AC converter: power flows in
from an isolated DC source and flows out as a shaped AC voltage.

\medskip
\textbf{Where exactly does cascading happen?}
In this paper, $n$ such cells are cascaded by connecting their \emph{AC output
terminals} in series --- the top terminal of cell~$i$ joins the bottom terminal
of cell~$(i{+}1)$, forming a single series string. The bottom terminal of the
lowest cell and the top terminal of the highest cell are the two ends of this
string. One end connects to the grid through filter inductance $L_f$; the other
end is the neutral. The same grid current $\ig$ is forced through every cell
in the string. The DC input ports play no part in the cascading --- they remain
fully isolated from one another throughout.

\medskip
The total AC voltage presented to the grid is the series sum of all cell outputs:
\[
    \vg(t) = v_1(t) + v_2(t) + \cdots + v_n(t)
\]

In sinusoidal PWM, each cell's transistors switch at a carrier frequency
$f_c \gg f$. The instantaneous output of cell $i$ is therefore a PWM waveform
that alternates rapidly between $+\Vdci{i}$, $0$, and $-\Vdci{i}$ --- not a
sinusoid. Averaged over one switching period, however, the \textbf{fundamental
component} of cell $i$'s output is:
\[
    \bar{v}_i(t) = \Vdci{i} \cdot \Mi \cdot \sin(\omg t)
\]

Summing across all $n$ cells, the fundamental component of the total output is:
\[
    \bar{\vg}(t) = \left(\sum_{i=1}^{n} \Vdci{i}\,\Mi\right)\sin(\omg t)
\]

\begin{warningbox}[Three Caveats on This Equation]
\begin{enumerate}
    \item \textbf{Fundamental only.} The actual instantaneous $\vg(t)$ contains
          high-frequency PWM harmonics on top of the sine. These are attenuated
          --- not eliminated --- by the filter inductance $L_f$.
    \item \textbf{No per-cell phase offset.} All cells in this paper share the
          same sinusoidal reference phase. There is no $\theta_i$ offset between
          cells in the voltage sum.
    \item \textbf{Linear modulation only.} The relation $\bar{v}_i = \Vdci{i}\Mi\sin(\omg t)$
          holds only for $\Mi \leq 1$ (non-clamped, linear region). Clamped
          modulation --- the key contribution of this paper --- modifies this,
          which is why later sections treat the clamped case separately.
\end{enumerate}
\end{warningbox}

\textbf{Why bother stacking?} Each cell's capacitor only needs to support a
fraction $\Vdci{i}$ of the total bus voltage, so low-voltage, low-cost
transistors suffice. A 540\,V grid interface is built from three 180\,V cells.
With sinusoidal PWM and phase-shifted carriers, $n$ cells together produce a
$(2n{+}1)$-level output; more levels mean lower harmonic distortion and a
smaller filter $L_f$ --- see \figref{fig:staircase_buildup}.

% TODO: Add a side-by-side n=1..4 comparison figure here (Figure 1).
%       Requirements: all panels on a normalised y-axis (v / n*Vdc) so the
%       shrinking orange error region is visually obvious across panels.
%       Defer until a clean Python figure meeting these criteria is ready.

\figentry{fig:staircase_buildup}{CHB_staircase_buildup}%
    {How three individual H-bridge cell outputs combine to form a
     seven-level multilevel waveform ($n = 3$, $f_c = 12f$, $M = 0.95$).
     \textbf{Rows 1--3:} each cell's unipolar PWM output (solid coloured
     line) is shown together with its triangular carrier (grey) and the
     shared sinusoidal reference $MV_{dc}\sin(\omega t)$ (dashed).
     Carriers are phase-shifted by $120^\circ$ so that the three cells
     switch at interleaved instants.
     \textbf{Row 4 (Sum):} the series sum $v_g = v_1 + v_2 + v_3$
     steps through seven discrete levels
     ($0,\,\pm V_{dc},\,\pm 2V_{dc},\,\pm 3V_{dc}$) and closely tracks
     the ideal sinusoid $3V_{dc}M\sin(\omega t)$ (red dashed line).
     The orange shading is the residual instantaneous error; the
     phase-shifted carriers ensure that no two cells switch at the same
     instant, spreading the transitions evenly and minimising the error.}{1.0}

\subsection{The Problem in One Sentence}

\textbf{Every cell shares the same AC current $\ig$, but each cell has its own
DC source delivering a different amount of power.}

If those source powers are unequal and nothing corrects it, the DC voltages
$\Vdci{i}$ drift apart. The cell with more input power charges up; the cell
with less power drains down. Left uncorrected, one cell charges to destruction
while another discharges to zero.

\medskip
This is the \emph{normal operating condition} for a CHB connected to:
\begin{itemize}
    \item \textbf{Photovoltaic strings}: different panels see different shading.
    \item \textbf{Battery packs}: cells age at different rates and reach different
          states of charge.
    \item \textbf{Power routing systems}: cells deliberately assigned unequal power
          to extend component lifetime.
\end{itemize}

Balancing is not an optional add-on --- it is a hard requirement for safe, stable
operation.

\subsection{What This Paper Proposes --- In One Paragraph}

The conventional fix is a per-cell PI controller that nudges each cell's
modulation index by a small $\Delta M_i$ each cycle. It works, but the measured
settling time is $\approx 1850$\,ms (92 grid cycles at 50\,Hz). The PI can
only make a small correction per cycle, otherwise it distorts the grid current.

\medskip
This paper asks a different question: \emph{what is the maximum loading power
this cell can absorb or deliver, given its modulation strategy?} It then
\textbf{directly sets} the modulation parameters $M_i$ and clamping angle
$\phic$ to deliver exactly that maximum loading power from cycle one.
The result: $\approx 140$\,ms settling time --- a factor of $13\times$ faster.

\begin{eqnbox}[The One Number That Matters]
\[
    \boxed{13\times \text{ faster:} \quad 1850\,\text{ms}
           \;\longrightarrow\; 140\,\text{ms}}
\]
Both methods reach the same steady-state result --- balanced DC-link voltages
and sinusoidal grid current. The only difference is \emph{how fast they get
there}.
\end{eqnbox}

\subsection{Prerequisites}

This document assumes you are comfortable with:
\begin{enumerate}
    \item \textbf{Basic switching converters}: duty cycle, average voltage, PWM.
          If you understand what a buck converter does, you are ready.
    \item \textbf{AC circuit analysis}: phasors, power factor, the formula
          $P = V I \cos\theta$.
    \item \textbf{PI controllers}: a proportional-integral controller integrates
          the error and adjusts its output. That is all.
\end{enumerate}

You do \emph{not} need: advanced state-space control theory, transistor-level
circuit simulation, or prior knowledge of CHB converters.

\subsection{Quick Symbol Reference}

Every symbol used in this document is defined here. Return to this table whenever
an equation introduces unfamiliar notation.

\begin{statebox}[Master Symbol Table]
\renewcommand{\arraystretch}{1.3}
\begin{tabular}{p{0.17\linewidth} p{0.40\linewidth} p{0.35\linewidth}}
\toprule
\textbf{Symbol} & \textbf{Meaning} & \textbf{Typical Value / Units} \\
\midrule
$n$                    & Number of H-bridge cells            & 3 (prototype), 4 (analysis) \\
$\Vdci{i}$            & DC-link voltage, cell $i$           & 180\,V (target each) \\
$\Vtot$                & Total DC voltage $\sum_i \Vdci{i}$ & $n \times 180$\,V \\
$\ig$, $\Ig$          & Grid current (inst.\ / amplitude)   & A; same through all cells \\
$\omg$                 & Grid angular frequency              & $2\pi \times 50$\,rad/s \\
$\phig$                & Grid power factor angle             & Fixed by load \\
$L_f$                  & Grid-side filter inductance         & mH \\
$\Mi$                  & Modulation index, cell $i$          & $0 \leq M_i \leq 1$ \\
$\phic$                & Clamping angle                      & $0$ (none) to $\pi/2$ (full) \\
$\PH$                  & Loading power (net into cap)        & W \\
$\PHi{i}$             & Loading power, cell $i$             & W \\
$\PHnc$                & Non-clamped loading power           & fn.\ of $\Mi$ only \\
$\PHcl$                & Clamped loading power               & fn.\ of $\Mi$ and $\phic$ \\
$P_{i,\mathrm{src}}$  & DC source power, cell $i$           & W; unequal between cells \\
\bottomrule
\end{tabular}
\end{statebox}

\clearpage

% ============================================================
\section{The Circuit --- What Is Actually Connected to What?}
\label{sec:circuit}
% ============================================================

Section~0 described the CHB in words. This section puts real components
on the page and names every wire, so that later equations have something
concrete to refer to.

% ── Fig 1 from the paper ──────────────────────────────────────────────────
\figentry{fig:paper_fig1}{DLVBMCHBC_Fig1_page01}%
    {Grid-connected CHB converter with unbalanced loading power
     $\PHi{1} \neq \PHi{2} \neq \PHi{3}$ (paper Fig.\,1).
     Each H-bridge cell has its own isolated DC-link capacitor fed by an
     independent DC source. The AC output terminals of all $n$ cells are
     connected in series; the series string drives the grid through
     filter inductance $L_f$.}{0.75}

\subsection{Reading the Circuit Diagram}

\figref{fig:paper_fig1} shows the complete system. Start at the bottom
and work upward.

\begin{circuitbox}[Component-by-Component Walk-through]
\begin{enumerate}
    \item \textbf{DC sources (right side of each cell).}
          Each cell $i$ has its own private DC source --- a PV string,
          battery pack, or rectifier --- delivering power $P_{i,\mathrm{src}}$
          and maintaining a DC-link capacitor at voltage $\Vdci{i}$.
          The dashed isolation boundary in the figure is not decorative:
          there is \emph{no} electrical path between the DC sides of
          different cells.

    \item \textbf{H-bridge switches (centre of each cell).}
          Four transistors per cell chop the DC voltage at carrier
          frequency $f_c$. The average AC voltage produced at the
          output terminals over one switching period is
          $\bar{v}_i = \Vdci{i}\Mi\sin(\omg t)$.

    \item \textbf{AC series string (left column).}
          The AC output terminals are stacked in series:
          the top terminal of cell~$i$ connects to the bottom terminal
          of cell~$(i{+}1)$. The total voltage across the string is
          $\vg = v_1 + v_2 + \cdots + v_n$.

    \item \textbf{Filter inductor $L_f$.}
          Sits between the AC string and the grid. It attenuates
          the high-frequency PWM harmonics and forces the grid
          current $\ig$ to be approximately sinusoidal.
          Crucially, \emph{the same} $\ig$ flows through every cell
          in the series string --- there is no way for current to
          take a different path.

    \item \textbf{Grid (far left).}
          Modelled as an ideal sinusoidal voltage source
          $v_{\mathrm{grid}}(t) = V_{\mathrm{grid}}\sin(\omg t + \phig)$.
          The grid sets the frequency $\omg$ and the power-factor
          angle $\phig$; the CHB controls the amplitude and phase of
          $\vg$ to regulate $\ig$.
\end{enumerate}
\end{circuitbox}

\subsection{The One Quantity That Drives Everything: Loading Power}
\label{subsec:loading_power}

Because the same current $\ig$ passes through every cell, each cell
\emph{must} process its share of the AC power whether it likes it or
not. But each cell also has its own DC source pushing power in from
the right. The net power that actually charges or discharges the
DC-link capacitor of cell $i$ is called its \textbf{loading power}:

\begin{eqnbox}[Definition of Loading Power]
\[
    \PHi{i} \;=\; \frac{1}{T}\int_{0}^{T} v_i(t)\,\ig(t)\,\mathrm{d}t
\]
where $T = 1/f$ is one grid period. This is the average AC power
\emph{delivered by cell $i$ to the grid} (positive = delivering,
negative = absorbing). In steady state, $\PHi{i}$ must equal the
DC source power $P_{i,\mathrm{src}}$ to keep $\Vdci{i}$ constant.
\end{eqnbox}

\begin{physbox}[Why ``Loading Power'' and Not Just ``Power''? --- The Governing Equation]
The DC-link capacitor of cell $i$ stores energy
$E_i = \tfrac{1}{2}C\Vdci{i}^2$.
Two power flows act on it simultaneously:
\begin{itemize}
    \item $P_{i,\mathrm{src}}$ flows \textbf{in} from the DC source.
    \item $\PHi{i}$ flows \textbf{out} to the AC series string.
\end{itemize}
The net rate of energy change is their difference, which by
$\tfrac{d}{dt}(\tfrac{1}{2}C V^2) = CV\dot{V}$ gives:
\[
    C\,\Vdci{i}\,\frac{d\Vdci{i}}{dt}
    \;=\; P_{i,\mathrm{src}} - \PHi{i}
\]
Three consequences follow directly from this one equation:
\begin{enumerate}
    \item \textbf{Steady state} ($\dot{V}_{dc,i} = 0$) requires exactly
          $\PHi{i} = P_{i,\mathrm{src}}$ --- the AC side must absorb
          precisely what the DC source supplies.
    \item \textbf{Overcharging:} if $P_{i,\mathrm{src}} > \PHi{i}$
          the RHS is positive $\Rightarrow$ $\Vdci{i}$ rises.
    \item \textbf{Discharging:} if $P_{i,\mathrm{src}} < \PHi{i}$
          the RHS is negative $\Rightarrow$ $\Vdci{i}$ falls.
\end{enumerate}
This is why the paper focuses entirely on controlling $\PHi{i}$:
it is the \emph{only} quantity the modulator can manipulate to
keep $\Vdci{i}$ at its target value.
\end{physbox}

\begin{warningbox}[The Key Asymmetry]
The grid current $\ig$ is \emph{shared} --- it is the same for every
cell. The DC source powers $P_{i,\mathrm{src}}$ are \emph{independent}
--- they depend on sunlight, battery state of charge, or deliberate
power routing and can differ arbitrarily between cells.
Consequently $\PHi{i}$ can be different for every cell, and the
DC-link voltages $\Vdci{i}$ will drift unless the controller
continuously corrects them. This is the problem the paper solves.
\end{warningbox}

\clearpage

\end{document}
