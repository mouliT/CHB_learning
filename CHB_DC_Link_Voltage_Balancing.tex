% ============================================================
%  DC-Link Voltage Balancing Modulation for Cascaded H-Bridge Converters
%  Paper 3 Reference Notes
%  Source: [Vasishta et al.], IEEE Access
%  Figures stored in: figures/
%  Naming convention: DLVBMCHBC_Fig[N]_page[XX].png  (caption-extracted at 200 DPI)
% ============================================================
\documentclass[12pt, a4paper]{article}

% ---- core packages ----
\usepackage[utf8]{inputenc}
\usepackage[T1]{fontenc}
\usepackage{amsmath, amssymb, amsthm}
\usepackage{bm}
\usepackage{siunitx}

% ---- figures ----
\usepackage{graphicx}
\graphicspath{{figures/}}
\usepackage{float}
\usepackage{caption}
\usepackage{subcaption}

% ---- layout ----
\usepackage[a4paper, top=2.5cm, bottom=2.5cm, left=3cm, right=2.8cm]{geometry}
\usepackage{parskip}
\setlength{\parskip}{6pt}
\setlength{\parindent}{0pt}

% ---- hyperlinks ----
\usepackage[colorlinks=true, linkcolor=blue!60!black, citecolor=blue!50!black]{hyperref}
\usepackage{bookmark}
\makeatletter
\pdfstringdefDisableCommands{%
    \def\vspace#1{}%
    \def\rule#1#2{}%
    \def\color#1{}%
    \let\leavevmode@ifvmode\relax%
    \def\,{ }%
}
\makeatother

% ---- styled boxes ----
\usepackage{tcolorbox}
\tcbuselibrary{skins, breakable}

% ---- tables ----
\usepackage{booktabs}
\usepackage{array}
\usepackage{longtable}

% ---- section formatting ----
\usepackage{titlesec}
\usepackage{xcolor}
\definecolor{chbblue}{RGB}{10, 55, 120}
\definecolor{chborange}{RGB}{190, 70, 5}
\definecolor{chbgreen}{RGB}{8, 95, 38}
\definecolor{chbred}{RGB}{170, 25, 25}
\definecolor{lightblue}{RGB}{216, 232, 252}
\definecolor{lightorange}{RGB}{255, 244, 220}
\definecolor{lightgreen}{RGB}{216, 247, 228}
\definecolor{lightred}{RGB}{255, 228, 228}
\definecolor{lightgray}{RGB}{245, 245, 245}

\titleformat{\section}{\large\bfseries\color{chbblue}}{{\thesection.}}{0.5em}{}
            [\vspace{-4pt}\rule{\textwidth}{0.5pt}]
\titleformat{\subsection}{\normalsize\bfseries\color{chbblue!80!black}}{{\thesubsection}}{0.5em}{}

% ---- tcolorbox styles ----
\newtcolorbox{circuitbox}[1][Circuit Description]{
    colback=lightblue, colframe=chbblue, coltitle=white,
    fonttitle=\bfseries\small, title={#1},
    breakable, enhanced, left=4pt, right=4pt, top=4pt, bottom=4pt}

\newtcolorbox{statebox}[1][State Variables \& Parameters]{
    colback=lightorange, colframe=chborange, coltitle=white,
    fonttitle=\bfseries\small, title={#1},
    breakable, enhanced, left=4pt, right=4pt, top=4pt, bottom=4pt}

\newtcolorbox{eqnbox}[1][Governing Equations]{
    colback=lightgreen, colframe=chbgreen, coltitle=white,
    fonttitle=\bfseries\small, title={#1},
    breakable, enhanced, left=4pt, right=4pt, top=4pt, bottom=4pt}

\newtcolorbox{physbox}[1][Physical Interpretation]{
    colback=lightgray, colframe=gray!60, coltitle=white,
    fonttitle=\bfseries\small, title={#1},
    breakable, enhanced, left=4pt, right=4pt, top=4pt, bottom=4pt}

\newtcolorbox{warningbox}[1][Watch Out]{
    colback=lightred, colframe=chbred, coltitle=white,
    fonttitle=\bfseries\small, title={#1},
    breakable, enhanced, left=4pt, right=4pt, top=4pt, bottom=4pt}

% ---- custom commands ----
\newcommand{\vg}{v_{g}}
\newcommand{\Vg}{V_{g}}
\newcommand{\ig}{i_{g}}
\newcommand{\Ig}{I_{g}}
\newcommand{\omg}{\omega}
\newcommand{\phig}{\phi_{g}}
\newcommand{\Vdc}{V_{dc}}
\newcommand{\Vdci}[1]{V_{dc,#1}}
\newcommand{\Vtot}{V_{dc,\mathrm{tot}}}
\newcommand{\Mi}{M_{i}}
\newcommand{\phic}{\varphi}
\newcommand{\vref}{v_{\mathrm{ref}}}
\newcommand{\PH}{P_{H}}
\newcommand{\PHi}[1]{P_{H,#1}}
\newcommand{\PHnc}{P_{H}^{\mathrm{nc}}}
\newcommand{\PHcl}{P_{H}^{\mathrm{cl}}}
\newcommand{\figref}[1]{Figure~\ref{#1}}

% ---- figure entry macro ----
\newcommand{\figentry}[4]{%
    \begin{figure}[H]
        \centering
        \includegraphics[width=#4\textwidth]{#2}
        \caption{#3}
        \label{#1}
    \end{figure}%
}

% ============================================================
\title{\textbf{DC-Link Voltage Balancing Modulation}\\[4pt]
       \textbf{for Cascaded H-Bridge Converters}\\[6pt]
       \large Paper 3 --- Circuit Topology, Control Architecture,\\
       \large Loading Power Theory, and Sorting Algorithm\\[4pt]
       \normalsize IEEE Access}
\date{\today}
% ============================================================

\begin{document}
\maketitle
\tableofcontents
\clearpage

% ============================================================
\setcounter{section}{-1}
\section{Before You Begin --- The Big Picture}
\label{sec:bigpicture}
% ============================================================

\subsection{What Is This Document?}

This document is a self-contained reference guide to the paper:

\begin{center}
\textit{``DC-Link Voltage Balancing Modulation for Cascaded H-Bridge Converters''}\\
Y.\,Ko, A.\,Tcai, and M.\,Liserre, \textit{IEEE Access}, 2021.
\end{center}

It walks through every figure, every equation, and every design decision in
Feynman style --- starting from the simplest possible picture and building up
to the full algorithm. No derivation is skipped; no equation is left unexplained.

\textbf{What you will understand by the end:}
\begin{itemize}
    \item Why a Cascaded H-Bridge (CHB) inverter needs DC-link voltage balancing.
    \item What ``loading power'' is and why it is the right quantity to control.
    \item Why clamped modulation is faster than conventional PI control for balancing.
    \item How to compute the exact modulation parameters to balance any imbalance
          in minimum time.
\end{itemize}

\subsection{The Simplest Possible Description of a CHB}

Start with a \textbf{single H-bridge}. It has four transistors arranged in a
bridge, a DC-link capacitor on one side, and an AC output on the other.
By switching the transistors at the right times, it converts DC to AC ---
a very fast ``polarity reversing switch'': at one instant the capacitor's
positive terminal connects to the AC output; at the next instant the negative
terminal does.

\medskip
\textbf{Now cascade $n$ of them in series.} Connect the AC outputs end-to-end,
like batteries stacked in a torch. Each cell produces $v_i(t)$ and the total
output is simply their sum:
\[
    v_{\mathrm{out}}(t) = v_1(t) + v_2(t) + \cdots + v_n(t)
\]

\textbf{Why bother stacking?} Each cell's capacitor only needs to hold a
fraction of the total bus voltage. A 540\,V AC bus can be built from three
cells with 180\,V capacitors, using cheap low-voltage transistors.
Modularity, fault tolerance, and near-sinusoidal output come for free.

\begin{physbox}[Analogy: The Multi-Battery Torch]
Imagine a torch with $n$ batteries in series. Each battery is one H-bridge cell.
The total voltage is $n \times V_{\rm battery}$. If all batteries are identical
and fully charged, the torch glows at full brightness. Now suppose battery~1 has
been partly discharged --- its voltage sags. The total output drops and the light
dims unevenly.

\medskip
DC-link balancing is the act of \emph{keeping all batteries at the same voltage
despite each one being charged by a different source at a different rate.}
\end{physbox}

\subsection{The Problem in One Sentence}

\textbf{Every cell shares the same AC current $\ig$, but each cell has its own
DC source delivering a different amount of power.}

If those source powers are unequal and nothing corrects it, the DC voltages
$\Vdci{i}$ drift apart. The cell with more input power charges up; the cell
with less power drains down. Left uncorrected, one cell charges to destruction
while another discharges to zero.

\medskip
This is the \emph{normal operating condition} for a CHB connected to:
\begin{itemize}
    \item \textbf{Photovoltaic strings}: different panels see different shading.
    \item \textbf{Battery packs}: cells age at different rates and reach different
          states of charge.
    \item \textbf{Power routing systems}: cells deliberately assigned unequal power
          to extend component lifetime.
\end{itemize}

Balancing is not an optional add-on --- it is a hard requirement for safe, stable
operation.

\subsection{What This Paper Proposes --- In One Paragraph}

The conventional fix is a per-cell PI controller that nudges each cell's
modulation index by a small $\Delta M_i$ each cycle. It works, but the measured
settling time is $\approx 1850$\,ms (92 grid cycles at 50\,Hz). The PI can
only make a small correction per cycle, otherwise it distorts the grid current.

\medskip
This paper asks a different question: \emph{what is the maximum loading power
this cell can absorb or deliver, given its modulation strategy?} It then
\textbf{directly sets} the modulation parameters $M_i$ and clamping angle
$\phic$ to deliver exactly that maximum loading power from cycle one.
The result: $\approx 140$\,ms settling time --- a factor of $13\times$ faster.

\begin{eqnbox}[The One Number That Matters]
\[
    \boxed{13\times \text{ faster:} \quad 1850\,\text{ms}
           \;\longrightarrow\; 140\,\text{ms}}
\]
Both methods reach the same steady-state result --- balanced DC-link voltages
and sinusoidal grid current. The only difference is \emph{how fast they get
there}.
\end{eqnbox}

\subsection{Prerequisites}

This document assumes you are comfortable with:
\begin{enumerate}
    \item \textbf{Basic switching converters}: duty cycle, average voltage, PWM.
          If you understand what a buck converter does, you are ready.
    \item \textbf{AC circuit analysis}: phasors, power factor, the formula
          $P = V I \cos\theta$.
    \item \textbf{PI controllers}: a proportional-integral controller integrates
          the error and adjusts its output. That is all.
\end{enumerate}

You do \emph{not} need: advanced state-space control theory, transistor-level
circuit simulation, or prior knowledge of CHB converters.

\subsection{Quick Symbol Reference}

Every symbol used in this document is defined here. Return to this table whenever
an equation introduces unfamiliar notation.

\begin{statebox}[Master Symbol Table]
\renewcommand{\arraystretch}{1.3}
\begin{tabular}{p{0.17\linewidth} p{0.40\linewidth} p{0.35\linewidth}}
\toprule
\textbf{Symbol} & \textbf{Meaning} & \textbf{Typical Value / Units} \\
\midrule
$n$                    & Number of H-bridge cells            & 3 (prototype), 4 (analysis) \\
$\Vdci{i}$            & DC-link voltage, cell $i$           & 180\,V (target each) \\
$\Vtot$                & Total DC voltage $\sum_i \Vdci{i}$ & $n \times 180$\,V \\
$\ig$, $\Ig$          & Grid current (inst.\ / amplitude)   & A; same through all cells \\
$\omg$                 & Grid angular frequency              & $2\pi \times 50$\,rad/s \\
$\phig$                & Grid power factor angle             & Fixed by load \\
$L_f$                  & Grid-side filter inductance         & mH \\
$\Mi$                  & Modulation index, cell $i$          & $0 \leq M_i \leq 1$ \\
$\phic$                & Clamping angle                      & $0$ (none) to $\pi/2$ (full) \\
$\PH$                  & Loading power (net into cap)        & W \\
$\PHi{i}$             & Loading power, cell $i$             & W \\
$\PHnc$                & Non-clamped loading power           & fn.\ of $\Mi$ only \\
$\PHcl$                & Clamped loading power               & fn.\ of $\Mi$ and $\phic$ \\
$P_{i,\mathrm{src}}$  & DC source power, cell $i$           & W; unequal between cells \\
\bottomrule
\end{tabular}
\end{statebox}

\clearpage

\end{document}
